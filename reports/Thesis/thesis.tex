\documentclass[12pt]{report}
\usepackage{helvet}
\usepackage[utf8]{vietnam}
\usepackage[a4paper, top=20mm, bottom=20mm, left=35mm, right=20mm]{geometry}
\usepackage{tikz}
\usepackage[nottoc,notlot,notlof]{tocbibind}
\usetikzlibrary{calc}
\usepackage[toc,page]{appendix}
\newcommand\HRule{\rule{\textwidth}{1pt}}
\usepackage{changepage}\usepackage{enumitem}
\renewcommand{\baselinestretch}{1.2}
\usepackage{fancyhdr}
\usepackage{hyperref}
\usepackage{bm}
\hypersetup{
    colorlinks=true,
    linkcolor=blue,
    filecolor=magenta,      
    urlcolor=cyan,
    bookmarks=true,
}
\usepackage{pgf,tikz,pgfplots}
\pgfplotsset{compat=1.13}
\usepackage{mathrsfs}
\usetikzlibrary{arrows}
\usepackage{float}
\newcommand{\degre}{\ensuremath{^\circ}}
\usepackage{amsmath}
\usepackage[noend]{algpseudocode}
\usepackage[ruled,linesnumbered,noresetcount]{algorithm2e}
\usepackage{multicol}
\usepackage{lipsum}
\usepackage{mwe}
\usepackage[lofdepth,lotdepth]{subfig}
\usepackage{caption}
\usepackage{epstopdf}
\usepackage[svgpath=figs/]{svg}
\usepackage{graphicx,import}
\usepackage{pdfpages}
\captionsetup[figure]{font=footnotesize}

\begin{document}

\begin{titlepage}
\begin{tikzpicture}[remember picture, overlay]
  \draw[line width = 0.5pt] ($(current page.north west) + (35mm,-20mm)$) rectangle ($(current page.south east) + (-20mm,20mm)$);
\end{tikzpicture}
\begin{center}
\fontsize{16pt}{12pt}\selectfont
TRƯỜNG ĐẠI HỌC BÁCH KHOA HÀ NỘI \\
VIỆN CÔNG NGHỆ THÔNG TIN VÀ TRUYỀN THÔNG 
\begin{tikzpicture}
\draw (5, 20) -- (10, 20);
\end{tikzpicture}
$ \ast $
\begin{tikzpicture}
\draw (5, 20) -- (10, 20);
\end{tikzpicture}
\\ [5cm]
\fontsize{30pt}{12pt}\selectfont
ĐỒ ÁN \\

\fontsize{32pt}{12pt}\selectfont
\textbf{TỐT NGHIỆP ĐẠI HỌC} \\ [1cm]

\fontsize{18pt}{12pt}\selectfont
\textsf{NGÀNH CÔNG NGHỆ THÔNG TIN}  \\ [2cm]

\fontsize{17pt}{12pt}\selectfont
\textbf{ĐỀ XUẤT GIAO THỨC ĐỊNH TUYẾN TRÁNH HỐ SỬ DỤNG HỌC TĂNG CƯỜNG TRONG MẠNG CẢM BIẾN KHÔNG DÂY} \\ [3cm]

\fontsize{14pt}{12pt}\selectfont
\begin{adjustwidth}{120pt}{}
Sinh viên thực hiện \hspace{9pt}:\quad\textbf{Lê Thị Khanh}\\
\end{adjustwidth}
\begin{adjustwidth}{269pt}{}
Lớp Việt Nhật IS1 - K59 \\
\end{adjustwidth}
\begin{adjustwidth}{120pt}{}
Giáo viên hướng dẫn \hspace{1pt}:\quad  \textbf{TS.}\\
\end{adjustwidth}
\begin{adjustwidth}{255pt}{}
\quad \textbf{Nguyễn Thanh Hùng} \\[2cm]
\end{adjustwidth} 

\fontsize{16pt}{12pt} Hà Nội, 05-2019
\end{center}
\end{titlepage}

\pagestyle{fancy}
\fancyhf{}
\fancyhead[R]{\chaptermark}
\fancyfoot[L]{\textit{Sinh viên thực hiện: Lê Thị Khanh 20142269 - K59 - Việt Nhật IS1}}
\fancyfoot[R]{\thepage}
\renewcommand{\footrulewidth}{0.5pt}
\renewcommand{\headrulewidth}{0pt}

\makeatletter
\let\ps@plain\ps@fancy
\makeatother

\newpage
\begin{center}
\LARGE \textbf{LỜI CÁM ƠN}
\end{center}
Khi những dòng cuối cùng của đồ án này được hoàn thành cũng là lúc em biết mình nên bày tỏ lòng biết ơn tới gia đình, thầy cô, bạn bè - những người đã luôn ở bên cạnh ủng hộ và cổ vũ em trong suốt thời gian qua.\\ \\
Trước tiên em xin được dành lời biết ơn sâu sắc nhất tới cô giáo, TS. \textbf{Nguyễn Phi Lê} - người đã luôn tận tình dẫn dắt em từ những bước đầu tiên đến với con đường nghiên cứu khoa học. Ở cô, em thấy được sự nhiệt huyết, tận tụy và cống hiến của một người làm nghiên cứu. Cô là một \textit{người thầy} mà mọi vấn đề khó khăn trong nghiên cứu của em đều có thể được giải quyết khi thảo luận cùng cô, không phải bằng cách áp đặt suy nghĩ của cô lên em mà bằng cách hướng em tới những tư duy học thuật đúng đắn, giúp em tự do khám phá và phát triển bản thân. Em cảm thấy mình may mắn vì đã gặp được một \textit{người thầy - người chị} như cô.\\ 
Lời tiếp theo, em xin chân thành cảm ơn thầy giáo, TS. \textbf{Nguyễn Thanh Hùng} đã luôn hướng dẫn, chỉ bảo và giúp đỡ em trong suốt quá trình làm đồ án. Sự quan tâm và theo dõi sát sao của thầy đã giúp em có thêm nhiều động lực để hoàn thành tốt đồ án này.\\ 
Người thầy cuối cùng mà em muốn bày tỏ lòng biết ơn chân thành, đó là thầy giáo, PGS.TS \textbf{Nguyễn Khanh Văn}. Tuy không trực tiếp hướng dẫn em, nhưng thầy đã cho em những lời khuyên vô cùng quý giá về phương hướng giải quyết vấn đề trong quá trình thực hiện đồ án.\\ \\
Xin cảm ơn tất cả những người bạn của tôi ở trường Đại học Bách Khoa Hà Nội, những người đã cho tôi những nụ cười, những ký ức thân yêu trong những năm tháng vừa qua.\\
Đặc biệt, tôi muốn cảm ơn \textbf{Chính, Hoàng, Nga, Việt} - những người bạn đã kề vai sát cánh cùng tôi trong suốt 5 năm đại học nhiều niềm vui nhưng cũng không thiếu những lúc khó khăn, những người bạn đã kéo tôi ra khỏi những giờ học mải miết để tận hưởng giây phút quý giá của tuổi thanh xuân. Họ là những người bạn mà tôi luôn trân trọng.\\ \\
Lời cuối cùng xin dành cho những người quan trọng nhất trong cuộc đời tôi. Gia đình là nơi cuộc sống bắt đầu và tình yêu không bao giờ kết thúc. Con xin cảm ơn gia đình đã luôn ở bên làm điểm tựa tinh thần vững chắc giúp con vượt qua khó khăn không chỉ trong quá trình thực hiện đồ án mà còn trên toàn bộ quá trình học tập, rèn luyện tại trường Đại học Bách khoa Hà Nội! 
\newpage
\begin{center}
\large \textbf{PHIẾU GIAO NHIỆM VỤ ĐỒ ÁN TỐT NGHIỆP}
\end{center}
\begin{enumerate}
\item \textbf{Thông tin về sinh viên} \\
Họ và tên sinh viên: Lê Thị Khanh \\
Điện thoại liên lạc: 0948 330 645 \hspace{50pt} Email: lekhanh.lamson@gmail.com \\
Lớp: Việt Nhật IS1 - K59 \hspace{86.5pt} Hệ đào tạo: Chính quy \\
Đồ án tốt nghiệp được thực hiện tại: Trường ĐHBKHN\\
Thời gian làm ĐATN: Từ ngày 16/10/2018 đến 24/05/2019
\item \textbf{Mục đích nội dung của ĐATN} \\
Đề xuất giao thức định tuyến tránh hố trong mạng cảm biến không dây sử dụng phương pháp Q-learning để đạt được hiệu quả cao về 3 yếu tố: tỷ lệ truyền tin thành công, độ trễ của gói tin, và thời gian sống của mạng.
\item \textbf{Các nhiệm vụ cụ thể của ĐATN}
\begin{itemize}
\item Nghiên cứu các giao thức định tuyến địa lý trong mạng cảm biến không dây đã được sử dụng trước đây.
\item Nghiên cứu về Q-learning và các giao thức định tuyến sử dụng Q-learning trong mạng cảm biến không dây.
\item Xây dựng giao thức định tuyến tránh hố sử dụng Q-learning trong mạng cảm biến không dây.
\item Cài đặt giao thức đề xuất và các giao thức so sánh vào công cụ mô phỏng NS2.
\end{itemize}
\item \textbf{Lời cam đoan của sinh viên}\\
Tôi - \textit{Lê Thị Khanh} - cam kết ĐATN là công trình nghiên cứu của bản thân tôi dưới sự hướng dẫn của \textit{TS. Nguyễn Thanh Hùng}.\\
Các kết quả nêu trong ĐATN là trung thực, không phải là sao chép toàn văn của bất kỳ công trình nào khác.\\
\begin{adjustwidth}{200pt}{}
\begin{center}
\textit{Hà Nội, ngày\qquad tháng\qquad năm 2019}\\
Tác giả ĐATN \\
\leavevmode\\
\leavevmode\\
\leavevmode\\
\textit{Lê Thị Khanh}
\end{center}
\end{adjustwidth} 
\pagebreak
\item \textbf{Xác nhận của giáo viên hướng dẫn về mức độ hoàn thành của ĐATN và cho phép bảo vệ }\\
\begin{adjustwidth}{200pt}{}
\begin{center}
\textit{Hà Nội, ngày\qquad tháng\qquad năm 2019}\\
Giáo viên hướng dẫn\\ 
\leavevmode\\
\leavevmode\\
\leavevmode\\
\textit{TS. Nguyễn Thanh Hùng}
\end{center}
\end{adjustwidth} 
\end{enumerate}

\newpage
\begin{center}
\large \textbf{TÓM TẮT NỘI DUNG ĐỒ ÁN TỐT NGHIỆP}\end{center}
Trong cuộc sống hiện đại, thay vì phải đối mặt với nạn đói, bệnh dịch, chiến tranh, chúng ta đang phải đối mặt với ô nhiễm môi trường và biến đổi khí hậu. Các kết luận về mức độ ô nhiễm không khí, hay sự thay đổi nồng độ khí nhà kính,... được đưa ra là nhờ một phần vào các trạm quan sát môi trường được triển khai từ mạng cảm biến không dây (Wireless Sensor Networks). Có thể nói các hệ thống mạng cảm biến không dây không hiện hữu trực tiếp và thân thuộc với chúng ta trong cuộc sống hằng ngày, nhưng chúng đang góp phần vào xây dựng và cải thiện cơ sở hạ tầng sống của con người. Bởi đặc điểm về bán kính truyền tin hẹp và tài nguyên hạn chế của các mạng cảm biến không dây, chúng cần một giao thức truyền tin nhanh chóng, đơn giản, và chính xác.\\
Trong khi các giao thức truyền tin trong trường hợp mạng bằng phẳng, nút mạng phân bố đều, không có vật cản đã đạt được kết quả khá tốt, thì các giao thức truyền tin trong trường hợp mạng có vật cản (như hồ nước, phương tiện giao thông...) tuy được đề xuất khá nhiều nhưng vẫn còn tồn tại các nhược điểm về: độ phức tạp tính toán, lượng thông tin cần lưu trữ và phát tán...\\
Trong đồ án này, chúng tôi sẽ đề xuất một giao thức truyền tin tránh vật cản (tránh hố) trong mạng cảm biến không dây - giao thức \textbf{\textit{HYBRID}}, là sự kết hợp giữa định tuyến địa lý và phương pháp học tăng cường, nhằm giải quyết các vấn đề còn tồn đọng về độ phức tạp tính toán, lượng thông tin cần lưu trữ và phát tán... của các giao thức định tuyến tránh hố đã được đề xuất trước đây.\\
Đồ án này bao gồm các nội dung: Nghiên cứu và phân tích các giao thức định tuyến tránh hố trước đây; Lý luận phương pháp xây dựng và đề xuất giao thức HYBRID; Trình bày kết quả thí nghiệm đánh giá hiệu năng của giao thức HYBRID; Kết luận và nêu hướng phát triển trong tương lai.
\newpage
\begin{center}
\large \textbf{ABSTRACT OF THESIS}
\end{center}
Recent years have witnessed the emergence of wireless sensor networks (WSNs) which
consist of tiny sensor nodes deployed over a region of intenest to monitor and control
the physical environment. Wireless sensor networks have been widely used in various
domains such as military target tracking and surveillance, natural disaster relief, agricultural and environmental monitoring, biomedical health monitoring, etc. In this thesis, we focus on large-scale WSNs which are based on the cooperation of a large number of sensor nodes. Typically, large-scale WSNs are used in monitoring applications such as agricultural monitoring, climate monitoring, forest monitoring, weather
monitoring, etc.\\
Due to the short-range communication nature of sensor nodes, routing becomes one of the most important issues which has received intensive research attention. As sensor nodes are equipped with only limited and non-rechargeable batteries, conserving energy consumption is an important factor in designing routing protocol. Energy
conservation can be achieved by shortening the routing path and reducing the overhead
caused by control packets. In many applications, the network can’t achieve its objective
if all the nodes can’t sense or report the sensed data, thus the death of even only one
node may cause the network to operate un-functionally. Accordingly, balancing traffic
over the network to extend the network lifetime is another important designing factor of
routing protocol in WSNs.\\
Geographic routing has been widely used in wireless sensor networks because of its
simplicity and efficiency. However, when subjected to routing holes (i.e., regions without sensor nodes that have communication capability), geographic routing suffers from the so called local minimum phenomenons. Many schemes was proposed to resolve these problems but all of the existing protocols cannot solve the load imbalance and routing path enlargement problems thoroughly.\\
In this thesis, we propose a novel protocol for bypassing hole in wireless sensor networks which can prolong the network's lifetime, ensure the successful packet delivery ratio, and shorten the end-to-end delay of data packet. The key ideas behind our approaches are the combination between geographic routing and machine learning (Q-learning in detail). Geographic routing method will helps data packets escape/bypass holes, while, machine learning could help sensor node to chose appropriate next hop in order to optimize the routing efficiency.
\pagestyle{fancy}
\fancyhf{}
\fancyfoot[L]{\textit{Sinh viên thực hiện: Lê Thị Khanh 20142269 - K59 - Việt Nhật IS}}
\fancyfoot[R]{\thepage}
\renewcommand{\footrulewidth}{0.5pt}

\tableofcontents
\listoffigures
\begingroup
\let\clearpage\relax
\listoftables
\listofalgorithms
\endgroup

\chapter{Giới thiệu}
\label{sec:1}
\section{Mạng cảm biến không dây}
\label{sec:1.1}
Mạng cảm biến không dây là một tập hợp gồm nhiều nút cảm biến (từ vài trăm tới vài nghìn nút) có thể kết nối không dây với nhau, được triển khai trong một vùng để thu thập thông tin của đối tượng cần khảo sát. Một nút cảm biến (còn được gọi là \textit{bộ cảm biến}, hay \textit{cảm biến}) là một thiết bị điện tử có khả năng cảm nhận những trạng thái hay quá trình vật lý hay hóa học từ môi trường, và biến đổi thành tín hiệu điện để thu thập thông tin về trạng thái hay quá trình đó (theo wikipedia \cite{wikisensor}). Thông tin này sau đó sẽ được truyền về trạm (base station) để xử lý dữ liệu. Bán kính cảm biến của một cảm biến khá thấp, thường không vượt quá 10m, và đồng thời bán kính truyền tin của nó cũng chỉ nằm trong khoảng từ vài chục tới vài trăm mét.\\
\begin{figure}[H]
\centering
\includegraphics[scale=0.5]{pics/wsn.eps}
\caption{Mạng cảm biến không dây}
\end{figure}
Mạng cảm biến không dây được ứng dụng khá rộng rãi trong thực tế: theo dõi mục tiêu trong quân sự \cite{military1,military2}, cứu trợ thiên tai \cite{disaster1, disaster2, disaster3, disaster4, disaster5}, giám sát trong nông nghiệp và môi trường \cite{agri1, agri2, agri3, agri4, agri5}, theo dõi sức khỏe \cite{health1, health2, health3}, etc.\\
Trong đồ án này, chúng tôi tập trung vào mạng cảm biến không dây có quy mô lớn (cỡ vài nghìn nút cảm biến). Chúng ta có thể bắt gặp các mô hình mạng cảm biến không dây quy mô lớn này trong các ứng dụng về giám sát trên diện rộng như: giám sát xâm nhập bất hợp pháp trong quân sự, đo độ ô nhiễm không khí, giám sát giao thông, đo độ ô nhiễm của sông hồ, etc.\\
Bắt đầu từ phần này chúng tôi sẽ dùng các thuật ngữ \textit{cảm biến, nút cảm biến, nút mạng, node}, hoặc \textit{bộ cảm biến} để chỉ một nút cảm biến trong mạng, và thuật ngữ \textit{hàng xóm}, hay \textit{1-hop neighbor} của một nút để chỉ những nút nằm trong bán kính truyền tin của nút đó. 

\section{Định tuyến địa lý và vấn đề hố định tuyến}
\label{sec:1.2}

\subsection{Định tuyến địa lý}
\label{sec:1.2.1}
Như đã nêu ở trên, sau khi các cảm biến thu thập thông tin từ môi trường, chúng cần truyền dữ liệu về trạm để xử lý. Với ba đặc trưng cơ bản của cảm biến là khả năng tính toán thấp, ít bộ nhớ và năng lượng không bền (theo \cite{agri1}, một cảm biến có không quá 640Kb bộ nhớ, và theo \cite{agri1, sensorenergy}, hầu hết các cảm biến đều có năng lượng không vượt quá 5V); việc chọn một giao thức định tuyến đơn giản và phù hợp để duy trì cho mạng hoạt động chính xác trong thời gian dài là vô cùng quan trọng.\\
Định tuyến địa lý là một trong những phương pháp định tuyến phổ biến nhất trong mạng cảm biến không dây bởi nó có tính đơn giản và hiệu quả. Định tuyến địa lý là giao thức định tuyến sử dụng thông tin tọa độ và ID của \textit{1-hop neighbor} (nút nằm trong bán kính truyền tin) để định tuyến, thay vì dùng bảng định tuyến như các mạng có dây hay mạng wifi. Một trong những giao thức định tuyến cổ điển đơn giản và phổ biến nhất trong nhóm giao thức định tuyến địa lý đó là \textit{giao thức \textbf{GREEDY}}, là giao thức mà gói tin luôn được gửi tới hàng xóm có khoảng cách tới đích gần nhất trong số tất cả các hàng xóm của nút hiện tại.
\begin{figure}[H]
\centering
\includegraphics[scale=0.5]{pics/geo_routing.eps}
\caption[Định tuyến địa lý]{Định tuyến địa lý. \textit{t} là nút đích, nút hiện tại $c_{1}$ chọn hàng xóm \textit{n} gần \textit{t} nhất để gửi gói tin tới.}
\end{figure}

\subsection{Hố định tuyến}
\label{sec:1.2.2}
\textit{\textbf{Hố định tuyến}} là vùng không tồn tại nút cảm biến, hoặc tồn tại nút cảm biến nhưng chúng bị chết khiến gói tin không thể đi qua được.\\
Giao thức greedy hoạt động khá tốt cho những mạng có mật độ nút mạng cao, tuy nhiên trong trường hợp mạng có mật độ thưa, hay có sự xuất hiện của hố định tuyến, giao thức này gặp phải hiện tượng \textit{\textbf{cực tiểu địa phương}} - là hiện tượng một nút mạng ở vị trí gần đích hơn tất cả các hàng xóm của nó. Lúc này giao thức greedy sẽ không truyền tin tiếp được nữa, và nút mạng đó được gọi là \textit{stuck node}.\\
Trong thực tế hố định tuyến xuất hiện khi mạng cảm biến không dây được triển khai ở những vùng có vật cản như ao, hồ,...; hoặc khi các nút mạng bị chết do: cạn kiệt năng lượng, cháy rừng, bị va chạm với sinh vật hay phương tiện khác dẫn tới bị hỏng, etc.\\
Nhìn chung, có hai hướng tiếp cận để giải quyết vấn đề hố định tuyến đó là: 1) đề xuất các phương pháp định tuyến tránh hố; 2) đưa sensor tới để lấp đầy hố định tuyến. Tuy nhiên thì phương pháp thứ 2 có phần tốn kém chi phí và trong trường hợp mạng gặp vật cản thì việc đặt sensor ở những vùng này cũng khó khăn hơn. Có khá nhiều giao thức truyền tin trong mạng có hố định tuyến (còn gọi là giao thức định tuyến tránh hố) đã được đề xuất, và chúng ta sẽ thảo luận ở chương \ref{sec:1.3} ngay tiếp theo đây.\\
Từ các chương sau, hố định tuyến sẽ được gọi tắt là \textit{hố}.

\section{Các hướng giải quyết bài toán định tuyến tránh hố}
\label{sec:1.3}
Có bốn cách tiếp cận phổ biến để giải quyết bài toán định tuyến tránh hố đó là:
\begin{itemize}
\item \textbf{Tiếp cận theo hướng đi men theo biên hố} \cite{boundhole, gpsr} gói tin sẽ được truyền đi theo giao thức greedy cho tới khi gặp hố. Khi gặp hố, gói tin sẽ đi men theo các nút nằm trên biên hố để tránh hố và đi tới đích. 
\item \textbf{Tiếp cận theo hướng xây dựng vùng cấm xung quanh hố} \cite{bedh, elipse, hexagon, octagon, bsmh}: trong các trường hợp đoạn thẳng nối nguồn và đích cắt hố, cách tiếp cận theo hướng đi men biên hố sẽ gặp phải hai tình trạng: các nút trên biên hố phải làm việc liên tục dẫn tới cạn kiệt năng lượng và chết đi, khiến hố bị lan rộng ra; và trong trường hợp biên hố rộng, ngoằn nghoèo phức tạp thì việc đi men biên hố sẽ kéo dài đường đi của gói tin, đồng thời làm tăng thời gian truyền tin từ nguồn tới đích, cũng làm tốn kém năng lượng của nhiều nút mạng hơn. Vì vậy nhiều cơ chế định tuyến dựa theo đa giác bao phủ quanh hố đã được đưa ra. Ý tưởng chung đó là dùng các hình dạng đa giác đơn giản như hình tròn, hình elip, hình lục giác, bát giác để bao phủ hố và trở thành vùng cấm. Thông tin về vùng cấm này sẽ được phát tán đi để những nút nhận được thông tin sẽ xây dựng đường định tuyến tránh hố. Ý tưởng này vừa rút ngắn đường đi của gói tin trong một số trường hợp, vừa tránh được tình trạng hố bị lan rộng.
\item \textbf{Tiếp cận theo hướng dựa vào kinh nghiệm định tuyến trong quá khứ} (heuristic routing) (\cite{edgr}): cơ chế định tuyến theo hướng này sẽ dùng thông tin định tuyến trong quá khứ để "học" về mô hình mạng, xây dựng các hàm để tối ưu hiệu quả truyền tin và xây dựng một số thông tin điều khiển cần thiết cho việc định tuyến ở các vùng quan trọng như khu vực xung quanh biên hố. Hướng tiếp cận này không cần biết trước thông tin về hố.
\end{itemize}
Như vậy, phương pháp định tuyến men theo biên hố sẽ gây ra tình trạng hố lan rộng và đường định tuyến bị kéo dài. \\ \\
Phương pháp định tuyến dựa trên đa giác bao phủ đã được đưa ra để giải quyết hai vấn đề này. Hướng tiếp cận này đã được chứng minh là khá hiệu quả trong việc đảm bảo cân bằng tải, kéo dài thời gian sống cho mạng và đảm bảo được cận trên của độ dài đường định tuyến, tuy nhiên lại không giải quyết trường hợp nút nguồn nằm trong vùng lõm của hố. 
Phương pháp định tuyến dựa trên các đường vị tự biên hố cũng cho hiệu quả rất tốt và giải quyết cả trường hợp nút nguồn nằm trong vùng lõm của hố. Tuy nhiên độ phức tạp tính toán đang còn cao dẫn tới thời gian gửi tin từ nguồn tới đích bị kéo dài.\\ \\
Phương pháp định tuyến heuristic không cần tốn nhiều năng lượng trong việc phát hiện hố và xây dựng các thông tin điều khiển, nhưng cũng vì đó mà các giao thức định tuyến heuristic tính đến thời điểm hiện tại chưa đạt hiệu quả cao. Gần đây, một số phương pháp định tuyến heuristic sử dụng học máy đã được đưa ra tuy nhiên chúng đều chưa giải quyết được vấn đề hố định tuyến.\\
Chi tiết về các hướng tiếp cận trên sẽ được trình bày ở chương \ref{sec:2.2}.

\section{Phương hướng và mục tiêu nghiên cứu của đồ án}
\label{sec:1.4}
Dựa vào các nhận xét đã nêu ra ở chương \ref{sec:1.3}, mục tiêu của chúng tôi là xây dựng một cơ chế định tuyến tránh hố đáp ứng các điều kiện:
\begin{enumerate}
\item Định tuyến được trong hai trường hợp nút nguồn nằm trong và ngoài vùng lõm của hố.
\item Hiệu quả thuật toán cao. Với hiệu quả của thuật toán được đánh giá bởi 3 tiêu chí: \textit{thời gian sống của mạng, thời gian truyền tin từ nguồn tới đích, tỉ lệ số gói tin được gửi thành công}.
\end{enumerate}
Để đáp ứng được hai điều kiện trên, chúng tôi đã chọn cách tiếp cận kết hợp giữa định tuyến địa lý và phương pháp học tăng cường, cụ thể là thuật toán Q-learning. Trong đó, định tuyến địa lý sẽ đảm bảo gói tin đi theo đúng hướng tới đích và tránh được hố, còn phương pháp học tăng cường nhằm mục tiêu đạt hiệu quả cao trong việc chọn đường đi tối ưu, tính toán nhẹ, và tiết kiệm bộ nhớ cho cảm biến.

\section{Các kết quả đạt được}
\label{sec:1.5}
Sau quá trình nghiên cứu và thực hiện, đồ án đã đạt được những kết quả như sau:
\begin{itemize}
\item Đã nghiên cứu, kế thừa ưu điểm và giải quyết thiếu sót của các giao thức định tuyến địa lý trong mạng cảm biến không dây đã được đề xuất trước đây.
\item Tìm hiểu và áp dụng được phương pháp học tăng cường vào trong định tuyến trong mạng cảm biến không dây.
\item Tìm hiểu và sử dụng thành thạo các công cụ mô phỏng mạng WiSSim \cite{wissim}, NS2 \cite{ns2}.
\item Triển khai hoàn thiện giao thức định tuyến sử dụng học tăng cường trong mạng cảm biến không dây và thu được kết quả mô phỏng tốt.
\item Có những kết luận, đánh giá kết quả đạt được của giao thức cũng như những vấn đề cần phải cải thiện trong tương lai.
\item Có hai bài báo được phát biểu tại các hội nghị quốc tế: 
\begin{itemize}
\item[$\star$] Phi-Le Nguyen, Yusheng Ji, Khanh Le, Thanh-Hung Nguyen, "\textbf{Load balanced and constant stretch routing in the vicinity of holes in WSNs}", 15th IEEE \textit{Annual Consumer Communications \& Networking Conference} (CCNC), 2018 \cite{vhr}.
\item[$\star$] Phi-Le Nguyen, Yusheng Ji, Khanh Le, Thanh-Hung Nguyen, "\textbf{Routing in the Vicinity of Multiple Holes in WSNs}", 5th \textit{International Conference on Information and Communication Technologies for Disaster Management} (ICT-DM), 2018 (\textbf{\textit{Best Student Paper Award}}) \cite{bsmh}.
\end{itemize}
\end{itemize}
\section{Cấu trúc đồ án}
\label{sec:1.6}
Ngoài phần Giới thiệu tổng quan (chương \ref{sec:1}), đồ án được chia thành bốn phần chính như sau:
\begin{itemize}
\item Chương \ref{sec:2}: Tổng hợp và trình bày về mặt ý tưởng của các giao thức định tuyến tránh hố trong mạng cảm biến không dây đã được đề xuất trước đây.
\item Chương \ref{sec:3}: Trình bày phương pháp học tăng cường và nếu một số giao thức sử dụng học tăng cường để định tuyến trong mạng cảm biến không dây.
\item Chương \ref{sec:4}: Trình bày giao thức định tuyến do chúng tôi đề xuất.
\item Chương \ref{sec:5}: Thí nghiệm và đánh giá hiệu năng của giao thức đề xuất.
\item Chương \ref{sec:6}: Kết luận.
\end{itemize}
\chapter{Các giao thức định tuyến tránh hố trong mạng cảm biến không dây}
\label{sec:2}

\section{Giao thức xác định hố}
\label{sec:2.1}
Trong một số giao thức định tuyến tránh hố trong mạng cảm biến không dây (GPSR \cite{gpsr}, EDGR \cite{edgr}, \textit{etc.}), gói tin có thể định tuyến mà không cần biết trước hình dạng của hố. Các giao thức này tuy có cách xác định đường đi đơn giản nhưng lại không mang lại hiệu quả cao về mặt cân bằng tải cũng như độ dài đường định tuyến. Việc xác định hố trước khi định tuyến giúp các nút mạng có được thông tin cần thiết để xây dựng các đường định tuyến mang lại hiệu quả cao. Giao thức định tuyến mà chúng tôi đề xuất trong đồ án này cũng sẽ xác định biên hố trước khi định tuyến.\\
Giao thức xác định hố trong mạng cảm biến không dây đầu tiên là giao thức  \textit{\textbf{BOUNDHOLE}} được đề xuất vào năm 2000, bởi Q.Fang \textit{et. al.} \cite{boundhole}. Giao thức này có đầu ra là danh sách các nút theo thứ tự liên tiếp nhau trên biên hố. Ngoài giao thức BOUNDHOLE, một số giao thức xác định hố khác cũng đã được đề xuất như: RollingBall \cite{rollingball}, thuật toán xác định biên hố Boundary Critical Point phân tán \cite{bcp}, thuật toán xác định biên hố phân tán Grid \cite{grid},... Dựa trên mức độ phổ biến và tính hiệu quả của các giao thức xác định biên hố, trong đồ án này chúng tôi lựa chọn sử dụng giao thức BOUNDHOLE. Sau đây sẽ là phần trình bày chi tiết về giao thức.\\
Giao thức BOUNDHOLE bao gồm 2 giai đoạn: giai đoạn 1 tất cả các nút cảm biến trong mạng sẽ kiểm tra xem nó có khả năng nằm trên biên hố hay không, đồng thời những nút có khả năng nằm trên biên hố cũng sẽ biết được hàng xóm nào có khả năng là nút liền kề nó trên biên hố; giai đoạn 2 những nút biết nó có khả năng nằm trên biên hố sẽ gửi đi gói tin xác định hố (Hole Boundary Detection - HBD) tới các nút liền kề mà nó đã xác định được ở giai đoạn 1. Khi gói tin HBD quay lại nút ban đầu đã tạo ra nó, danh sách tất cả các nút trên biên hố sẽ được xác định.\\
Ở giai đoạn 1, các nút mạng sẽ kiểm tra mình có nằm trên biên hố không nhờ luật TENT. Tác giả đã đưa ra hai khái niệm về \textit{Strongly Stuck Node} và \textit{luật TENT} như sau:
\begin{enumerate}
\item \textit{\textbf{Strongly Stuck Node}}
Một nút mạng \textit{p} được gọi là \textit{strongly stuck node} nếu tồn tại một điểm \textit{q} nằm ngoài vùng bán kính phủ sóng của p sao cho không có hàng xóm nào của \textit{p} gần \textit{q} hơn chính \textit{p}. Nút mạng nào là \textit{strongly stuck node} thì sẽ có khả năng nằm trên biên hố.
\begin{figure}[H]
\centering
\includegraphics[scale=0.6]{pics/stuck_node_2.png}
\caption{Strongly stuck node}
\end{figure}

\item \textit{\textbf{Luật TENT}} Luật TENT được dùng để kiểm tra xem một nút mạng có phải là \textit{strongly stuck node} hay không.\\
Với mỗi nút mạng \textit{p}, trước tiên tác giả sắp xếp các hàng xóm của \textit{p} theo thứ ngược chiều kim đồng hồ. Và các góc được đề cập ở đây được tính theo ngược chiều kim đồng hồ. Ký hiệu $B_{r}(x)$ là đường tròn tâm \textit{x} bán kính \textit{r}. Gọi $u, v$ là hai nút hàng xóm của \textit{p}. Vẽ 2 đường trung trực của \textit{up} và \textit{vp}, \textit{$l_{1}$, $l_{2}$}. \textit{$l_{1}$, $l_{2}$} cắt nhau tại \textit{o} và chia mặt phẳng thành 4 phần (hình 4.2). Chỉ những điểm nằm trong góc chứa \textit{p} mới gần \textit{p} hơn \textit{u} và \textit{v}. Nếu \textit{o} nằm trong vùng bán kính truyền tin của \textit{p}, thì với bất kỳ điểm \textit{q} nào nằm trong góc \textit{vpu} và nằm ngoài bán kính phủ sóng của \textit{p}, một trong hai điểm \textit{u} hoặc \textit{v} sẽ gần \textit{q} hơn \textit{p}, khi đó \textit{p} không phải là strongly stuck node. Nói cách khác, \textit{p} là strongly stuck node khi \textit{o} nằm ngoài bán kính phủ sóng của \textit{p}. Nhờ luật TENT, những hàng xóm có khả năng nằm liền kề \textit{p} trên biên hố (\textit{u, v}) cũng được đánh dấu. Góc \textit{vpu} được gọi là \textit{stuck angle}.
\end{enumerate}
Sau giai đoạn 1, các strongly stuck node và các stuck angle đã được đánh dấu. Mỗi strongly stuck node sẽ tạo một gói tin HBD (Hole Boundary Detection) để xác định biên hố. Gói tin HBD được gửi đi lần lượt tới hàng xóm nằm trong stuck angle của strongly stuck node, theo ngược chiều kim đồng hồ. Mỗi khi đi tới 1 strongly stuck node, gói tin HBD sẽ thêm ID và tọa độ của nút đó vào trong dữ liệu của mình. Khi HBD quay lại nút ban đầu đã tạo ra nó, chúng ta thu được một danh sách các nút trên biên hố theo thứ tự liền kề ngược chiều kim đồng hồ. Để tránh tất cả các strongly stuck node trên biên hố đều gửi gói tin HBD đi một vòng xung quanh hố (gây lãng phí năng lượng), mỗi strongly stuck node khi nhận được gói tin HBD xuất phát từ nút có tung độ thấp hơn nó, sẽ hủy gói tin HBD đi. Cuối cùng, chỉ có duy nhất một strongly stuck node với tung độ thấp nhất sẽ nhận lại được đầy đủ danh sách các nút trên biên hố.
\section{Giao thức định tuyến tránh hố}
\label{sec:2.2}

\subsection{Tiếp cận theo hướng đi men biên hố}
\label{sec:2.2.1}
Để tránh hố, các giao thức định tuyến cổ điển tiếp cận theo hướng sử dụng định tuyến tham lam ở những vùng có mật độ nút mạng dày (có thể tìm được hàng xóm gần đích hơn nút hiện tại), và chuyển sang định tuyến men biên hố (perimeter fowarding) khi gặp hố. Khi định tuyến men biên hố, các nút mạng cần xây dựng một đồ thị gọi là \textit{planar graph}, là đồ thị có các cạnh chỉ cắt nhau tại các đầu mút của chúng. Cơ chế định tuyến men biên hố sử dụng planar graph để tìm nút tiếp theo gói tin có thể được chuyển tới mà đường đi không cắt hố. Giao thức GPSR (Greedy Perimeter Stateless Routing) được đề xuất bởi Brad Karp \textit{et. al.} vào năm 2000 là giao thức được biến đến nhiều nhất trong tất cả các giao thức tiếp cận theo hướng đi men theo biên hố. Các giao thức tiếp cận theo hướng đi men theo biên hố có thể được tìm thấy ở \cite{}. Mặc dù các phương pháp tiếp cận này có thể giải quyết được hiện tượng cực tiểu địa phương, nhưng chúng lại gặp phải 2 vấn đề quan trọng. Vấn đề thứ nhất là hiện tượng đường đi bị kéo dài.... Vấn đề thứ hai là đường định tuyến sẽ chỉ tập trung ở biên hố khiến cho cân bằng tải của mạng không tốt. Subarmanian et al. \cite{subar} đã chỉ ra rằng việc định tuyến men hố này làm giảm đáng kể thông lượng của toàn mạng bởi vì đường truyền tập trung ở biên hố.
\begin{figure}[H]
\centering
\includegraphics[scale=0.5]{pics/traditional_routing.eps}
\caption{Định tuyến men biên hố}
\end{figure}
\subsection{Tiếp cận theo hướng xây dựng vùng cấm xung quanh hố}
Để giải quyết 2 vấn để của phương pháp định tuyến men biên hố cổ điển, nhiều giao thức định tuyến tiếp cận theo hướng xây dựng vùng cấm xung quanh hố đã được đề xuất như:  \textit{BEDH} \cite{bedh}, \textit{HEXAGON} \cite{hexagon}, \textit{OCTAGON} \cite{octagon}, etc. Các giao thức này đều dùng một hình hình học đơn giản (hình tròn \cite{bedh}, elipse \cite{elipse}, lục giác \cite{hexagon}, bát giác \cite{octagon},...) để bao hố và làm vùng cấm không cho gói tin đi vào bên trong. Cách làm này nhằm mục đích giảm tải ở biên hố, và rút ngắn đường định tuyến trong trường hợp hố có nhiều vùng lõm phức tạp.
\begin{figure}[H]
\centering
\subfloat[][Vùng cấm hình tròn]{
\includegraphics[width=0.3\textwidth, height=0.3\textwidth]{pics/circle.eps} 
\label{fig:subfig1}}
\qquad
\subfloat[][Vùng cấm hình elipse]{
\includegraphics[width=0.4\textwidth, height=0.3\textwidth]{pics/ellipse.eps}
\label{fig:subfig2}}
\caption{Các giao thức định tuyến xây dựng vùng cấm quanh hố}
\label{fig:globfig}
\end{figure}
Tuy vậy, một số giao thức như BEDH \cite{bedh}, ELIPSE \cite{elipse},... vẫn chưa giải quyết triệt để được vấn đề đường đi bị kéo dài (Hình 2.3). Và hầu hết tất cả các giao thức (BEDH \cite{bedh}, ELIPSE \cite{elipse}, HEXAGON \cite{hexagon},...) cũng chưa giải quyết được triệt để vấn đề tập trung tải ở biên hố. Bởi thay vì tập trung tải ở biên hố, các giao thức này lại khiến đường định tuyến tập trung ở vùng biên của vùng cấm.\\
Khác với các giao thức trên, giao thức OCTAGON \cite{octagon} không những cho gói tin dữ liệu đi men theo vùng cấm có dạng bát giác, mà còn vị tự bát giác này ra thành những bát giác có kích thước lớn hơn, để gói tin dữ liệu đi men theo những bát giác vị tự ấy. Tỷ lệ vị tự khác nhau cho mỗi lần truyền gói tin dữ liệu sẽ đảm bảo được sự khác nhau của đường định tuyến. Cách làm này đã giải quyết được triệt để 2 vấn đề về độ dài đường định tuyến và cân bằng tải nêu trên.
Ngoài ra, gần đây tác giả Phi Lê Nguyễn \textit{et. al.} đã đề xuất giao thức định tuyến đảm bảo cân bằng tải và độ dài đường định tuyến không những cho trường hợp nút nguồn ở ngoài hố mà còn giải quyết cả các trường hợp nút nguồn ở bên trong hố \cite{bsmh}. Tác giả đã xác định một tập các đường đi Ơ-clit vượt hố, vị tự các đường Ơ-clit ấy, rồi chọn ra một đường đi ngẫu nhiên trong số ấy. Đường đi này chứa các đích tạm thời gọi là anchor point, gói tin sẽ đi theo định tuyến tham lam tới các anchor point ấy cho tới khi tới được đích (Hình 2.4). Bằng chứng minh lý thuyết và thực nghiệm, tác giả đã chứng minh được tính hiệu quả về mặt cân bằng tải và tỷ lệ của độ đường đi thực tế so với đường đi Ơ-clit ngắn nhất không vượt quá $1 + \epsilon$ với $\epsilon$ là hằng số xác định trước.\\
Tuy nhiên, tất cả các giao thức đi theo hướng tiếp cận này đều cần phải phát tán thông tin về hố, thông tin về đa giác bao phủ của hố tới các nút mạng nằm xa hố, để đảm bảo đường đi được xây dựng một cách chính xác. Việc làm này làm tiêu hao năng lượng của các cảm biến. 
\begin{figure}[H]
\centering
\includegraphics[width=\textwidth]{pics/routing1.eps}
\caption[Đường định tuyến của giao thức BSMH]{Đường định tuyến của giao thức BSMH. Đường màu xanh da trời là đường Ơ-clit vượt hố cơ bản, đường màu đỏ là đường vị tự của đường cơ bản và cũng chính là đường chứa các anchor point trên thực tế.}
\end{figure}
\label{sec:2.2.2}

\subsection{Tiếp cận theo hướng heuristic}
\label{sec:2.2.4}
Có khá nhiều giao thức định tuyến sử dụng phương pháp heuristic để đảm bảo gói tin tránh được hố và đảm bảo cân bằng tải cho mạng. Yu \textit{et al.} \cite{holeplastic} đã đề xuất một cơ chế để tránh hiện tượng cực tiểu địa phương bằng cách xác định các nút nằm bên trong phần lõm của hố và không cho chúng tham gia vào quá trình định tuyến. Giao thức này chỉ cần ít năng lượng để phát tán thông tin về hố, và cũng tránh được hiện tượng gói tin đi đâm vào hố nhưng nó vẫn vướng phải một vấn đề quan trọng đó là cân bằng tải kém, bởi gói tin chỉ đi men theo phần biên lồi của hố, và đường đi của các gói tin từ cùng một nguồn là luôn giống nhau.
\begin{figure}[H]
\centering
\includegraphics[scale=0.4]{pics/edgr.png}
\caption{Giao thức EDGR}
\end{figure}
Trong \cite{edgr}, Haojun Huang \textit{et al.} đã đề xuất phương pháp định tuyến trong đó, một vài gói tin "khám phá" sẽ được gửi đi từ nguồn tới đích để xây dựng 2 đường đi cơ bản theo 2 hướng bên trái hoặc bên phải của hố theo hướng từ nguồn tới đích. Hai đường đi cơ bản này tương tự với đường đi Ơ-clit ngắn nhất mà không cắt bao lồi của hố. Chúng bao gồm các đích tạm thời, sau đấy các nút mạng sẽ chọn next hop để truyền gói tin dữ liệu đi lần lượt qua các đích tạm thời này sao cho một điều kiện về năng lượng giữa các nút hàng xóm với nhau được thỏa mãn. Điều kiện về năng lượng này nhằm mục đích cân bằng giữa năng lượng và tần suất gửi tin của các nút mạng, đảm bảo cân bằng tải cho mạng. Tuy vậy, cách giải quyết này lại không đưa ra chứng minh hoặc cơ sở lý thuyết để đảm bảo về độ dài đường định tuyến hay sự đa dạng của đường đi, nhất là trong các trường hợp nút nguồn nằm bên trong vùng lõm của hố.\\ \\ 
Dựa trên những ưu và nhược điểm của các giao thức định tuyến địa lý đã được phân tích ở trên, trong đồ án này chúng tôi sẽ đề xuất một giao thức định tuyến tránh hố - giao thức \textbf{\textit{HYBRID}}, kết hợp giữa định tuyến địa lý và phương pháp học tăng cường (cụ thể là thuật toán Q-learning).\\ \\ 
Ý tưởng của chúng tôi như sau:
\begin{itemize}
\item Để tiết kiệm năng lượng khi phát tán thông tin hố, chúng tôi sẽ chỉ phát tán thông tin chính xác của hố tới các nút nằm bên trong biên hố. Những nút nằm bên ngoài biên hố sẽ được học thông tin về hố nhờ quá trình back propagation của thuật toán Q-learning.
\item Đường định tuyến của các thuật toán trước đây hoặc là sẽ đi men theo biên hố hoặc là sẽ đi dựa trên các đích tạm thời của đường đi được xác định trước. Cách làm thứ nhất không đảm bảo được cân bằng tải cho mạng. Cách làm thứ hai đòi hỏi tính toán nhiều gây tốn thời gian và năng lượng của các nút mạng. Đường định tuyến theo thuật toán HYBRID của chúng tôi sẽ chỉ tính toán ngay tại thời điểm truyền tin dựa trên thông tin của hàng xóm, trong khi vẫn đảm bảo được cân bằng tải của mạng và độ dài của đường đi nhờ vào việc sử dụng Q-learning để cân bằng các yếu tố về năng lượng còn lại, khoảng cách tới đích, khoảng cách tới hố, ... của các nút mạng.
\end{itemize}
Nguyên lý về phương pháp học tăng cường và cụ thể là thuật toán Q-learning sẽ được trình bày ở ngay phần tiếp theo dưới đây.
\chapter{Phương pháp học tăng cường}
\label{sec:3}

\section{Lý thuyết và ứng dụng của phương pháp học tăng cường}
\label{sec:3.1}
\textit{Học tăng cường (reinforcement learning)} là một trong số các phương pháp học máy, bên cạnh \textit{học có giám sát}, \textit{học bán giám sát} và \textit{học không có giám sát} (phân nhóm dựa trên phương thức học). Học tăng cường là cách thức một đối tượng (\textit{agent}) trong một môi trường (\textit{environment}) chọn cách hành động (\textit{action}) để đạt được hiệu quả tối ưu (hiệu quả tối ưu đối với một mục đích được xác định trước), dựa vào kinh nghiệm hành động từ quá khứ. Phương pháp này là quá trình: "đưa ra hành động - nhận lại phần thưởng từ môi trường - đưa ra hoạt động tối ưu hơn" liên tiếp của đối tượng (Hình 3.1). Phần thưởng (\textit{reward)} là một hàm số có giá trị vô hướng, được tính dựa trên các yếu tố mà đối tượng muốn học. Một cách hình thức, một model học tăng cường bao gồm:
\begin{itemize}
\item S: tập các trạng thái (\textit{state}) của môi trường;
\item A: tập các hành động (\textit{action});
\item R: tập các khoản thưởng (\textit{reward}) với giá trị vô hướng;
\end{itemize}
\begin{figure}[H]
\centering 
\includegraphics[scale=0.4]{pics/reinforcement_learning.png}
\caption{Quá trình học tăng cường}
\end{figure}
Tại mỗi thời điểm t, đối tượng thấy được trạng thái của nó là $s_{t} \in S$ và tập các hành động có thể $A(s_{t})$. Nó chọn một hành động $a \in A(s_{t})$ và nhận được từ môi trường trạng thái mới $s_{t+1}$ và một khoản thưởng $r_{t+1}$. Dựa trên các tương tác này, đối tượng học tăng cường phải phát triển một chiến lược $\pi : S \rightarrow A$ có tác dụng cực đại hóa lượng $R = r_{0} + r_{1} + ... + r_{n}$ với các quá trình học có trạng thái kết thúc, hoặc lượng $R = \sum_{t}\gamma^{t}r_{t}$ với các quá trình học không có trạng thái kết thúc (trong đó $\gamma$ là một hệ số giảm khoản "thưởng trong tương lai" nào đó, với giá trị trong khoảng 0.0 và 1.0).\\
Do đó, học tăng cường thích hợp cho các bài toán có quá trình học lặp lại dài hạn, cần cân bằng được mất giữa các khoản thưởng ngắn hạn và dài hạn. Học tăng cường đã được áp dụng thành công cho nhiều bài toán, trong đó có điều khiển robot, điều vận thang máy, viễn thông, game atari, game mario, cờ vây, cờ vua, ect.
\section{Giới thiệu các thuật toán học tăng cường}
\label{sec:3.2}
Trong cuốn sách \textit{Reinforcement Learning: An Introduction} của Richard S. Sutton và Andrew G. Barto, tác giả chia các phương pháp học tăng cường thành hai nhóm dựa theo không gian trạng thái và hành động của đối tượng:
\begin{itemize}
\item \textbf{\textit{Tabular Solution Methods}}: là phương pháp học cho các đối tượng có không gian tập trạng thái và tập hành động đủ nhỏ để các giá trị đánh giá độ tốt của các hành động có thể biểu diễn dưới dạng bảng. Trong nhóm này, các phương pháp có thể tìm ra chính xác giá trị tối ưu nhất mỗi khi thực hiện chiến lược để quyết định hành động tiếp theo.
\item \textbf{\textit{Approximate Solution Methods}}: là phương pháp học cho các đối tượng có không gian tập trạng thái và tập hành động lớn, bởi sự giới hạn về khả năng tính toán, agent chỉ có thể tìm được xấp xỉ giá trị tối ưu mỗi khi thực hiện chiến lược để quyết định hành động tiếp theo.
\end{itemize}
Trong đồ án này, chúng tôi lựa chọn sử dụng nhóm phương pháp \textit{Tabular solution methods} bởi lý do: các nút cảm biến chỉ có thể gửi gói tin tới các hàng xóm của nó vì vậy không gian tập hành động của mỗi trạng thái là nhỏ (trong thực thế, mỗi cảm biến trong mạng cảm biến không dây có từ $5 \rightarrow 20$ hàng xóm). \\
Một số phương pháp học tabular solution method phải kể đến như: Multi-armed Bandits, Finite Markov Decision Processes, Dynamic Programming, Monte Carlo Methods, Temporal-Difference Learning... 
\section{Thuật toán Q-learning}
\label{sec:3.3}
\textit{\textbf{Q-learning}} là một trong những thuật toán model-free phổ biến nhất của nhóm thuật toán tabular solution methods. Phương pháp này sử dụng một ma trận $Q(s, a)$ của các trạng thái và hành động, mỗi một cặp (\textit{trạng thái, hành động}) sẽ có một giá trị gọi là $Q_{value}$ thể hiện độ tốt của việc thực hiện hành động \textit{a} khi đối tượng đang ở trạng thái \textit{s}. Giá trị $Q_{value}$ được cập nhật bởi công thức sau:
\begin{equation}
Q(s,a) \leftarrow (1 - \alpha). Q(s,a) + \alpha . [r + \gamma . max_{a'}Q(s', a')]
\end{equation}  
Trong đó: $s'$ là trạng ngay tiếp theo của trạng thái s, $\alpha$ là \textit{\textbf{learning rate}}, r là \textbf{\textit{reward}} và $\gamma$ là \textbf{\textit{discount factor}}, $max_{a'}Q(s', a')$ là giá trị $Q_{value}$ lớn nhất khi đối tượng đang ở trạng thái $s'$, được gọi là V-value của $s'$. Quá trình học của thuật toán Q-learning được thể hiện ở sơ đồ dưới đây.
\begin{figure}[H]
\centering 
\includegraphics[scale=0.3]{pics/q_learning.jpg}
\caption{Quá trình học của phương pháp Q-learning}
\end{figure}
\section{Các giao thức định tuyến sử dụng học tăng cường trong mạng cảm biến không dây}
\label{sec:3.4}
Một số giao thức định tuyến trong mạng cảm biến không dây sử dụng học tăng cường đã được đề xuất, hầu hết các tác giả đều sử dụng phương pháp Q-learning, như: \textit{Adaptive Routing} \cite{adaptiveqlearning},  \textit{Reinforcement Learning Based Geographic Routing} \cite{qlearningwsn}, \textit{QELAR} \cite{qelar}. Trong các giao thức này, tác giả định nghĩa tập các trạng thái, hành động, môi trường của mô hình học tăng cường như sau:
\begin{itemize}
\item \textbf{\textit{agent}}: là các gói tin dữ liệu
\item \textbf{\textit{enviroment}}: là tập hợp tất cả các nút mạng
\item \textbf{\textit{state s}}: là trạng thái gói tin đi tới một nút \textbf{\textit{s}} trong mạng
\item \textbf{\textit{action a}}: là sự kiện nút \textit{\textbf{a}} được chọn làm next hop để gửi gói tin
\end{itemize} 
Sau mỗi lần gói tin được gửi đi, $Q_{value}$ của hàng xóm mà nó vừa gửi gói tin tới sẽ được cập nhật theo công thức 3.1. Công thức 3.1 là công thức chung cho tất cả các thuật toán Q-learning, yếu tố quyết định tính hiệu quả giữa các thuật toán chính là cách xây dựng hàm tính reward. Trong các thuật toán định tuyến sử dụng Q-learning đã được đề xuất, tác giả xây dựng hàm tính reward dựa trên 2 yếu tố: năng lượng còn lại và khoảng cách tới đích của các hàng xóm, và không giải quyết trường hợp mạng có hố. Tuy giao thức được đề xuất trong \cite{qlearningwsn} có đề cập tới việc giải quyết hiện tượng cực tiểu địa phương dựa vào việc drop gói tin và gửi lại một reward rất thấp mỗi khi gói tin gặp stuck node, cách làm này chỉ hiệu quả khi mạng chỉ tồn tại một vài vùng có mật độ sensor thấp khiến cho gói tin đi vào đó gặp phải hiện tượng cực tiểu địa phương, còn trong trường hợp mạng có hố rộng và phức tạp, giao thức này phải drop rất nhiều gói tin trước khi tìm được đường định tuyến tránh hố hợp lý, dẫn đến tỷ lệ truyền tin thành công thấp.\\
Qua việc tìm hiểu về học tăng cường nói chung và các phương pháp học máy nói riêng, nguyên nhân chúng tôi lựa chọn áp dụng học tăng cường vào giao thức định tuyến có thể được lý giải như sau:
\begin{itemize}
\item[-] Học tăng cường thường không đòi hỏi tính toán phức tạp. Điều này phù hợp với sự hạn chế về tài nguyên của sensor. 
\item[-] Học tăng cường là mô hình vừa học vừa tối ưu hành động dựa trên kinh nghiệm của quá khứ, nó cho phép chúng ta cập nhật mô hình real time và không yêu cầu một tập dữ liệu lớn ngay từ đầu. Điều này phù hợp với tính chất biến đổi liên tục của mạng cảm biến không dây.
\end{itemize}
Trong các phương pháp học tăng cường, chúng tôi lựa chọn sử dụng Q-learning bởi là một phương pháp học model-free đơn giản, dựa trên kinh nghiệm thử và sai, không cần lưu trữ quá nhiều thông tin, hay tính toán quá phức tạp.\\ \\
Sau đây, chúng tôi sẽ đề xuất một giao thức định tuyến trong mạng cảm biến không dây có hố: \textit{giao thức \textbf{HYBRID}}. Giao thức này là sự kết hợp giữa định tuyến địa lý và phương pháp Q-learning. Bằng thực nghiệm, chúng tôi sẽ chứng minh rằng giao thức đề xuất đảm bảo được 3 yếu tố sau: 
\begin{itemize}
\item \textit{Tỷ lệ truyền tin thành công cao} (trên 99\%).
\item \textit{Độ trễ của gói tin thấp.} 
\item \textit{Thời gian sống của mạng được kéo dài hơn so với các giao thức trước đây.}
\end{itemize}
\chapter{Đề xuất giao thức định tuyến tránh hố trong mạng cảm biến không dây sử dụng phương pháp học tăng cường}
\label{sec:4}
Trong chương này, chúng tôi sẽ trình bày giao thức định tuyến sử dụng phương pháp học tăng cường để tìm đường đi trong các mạng cảm biến không dây có sự tồn tại của hố. Như đã đề cập ở các phần trước, mục tiêu xây dựng giao thức của chúng tôi là kéo dài thời gian sống cho mạng. Thời gian sống của mạng được tính từ thời điểm mạng bắt đầu hoạt động cho tới khi nút đầu tiên chết. Giao thức được đề xuất dưới đây sẽ đảm bảo các yếu tố sau:
\begin{itemize}
\item Đảm bảo cân bằng tải cho mạng để tránh tình trạng một nút phải làm việc quá tải dẫn đến hết năng lượng nhanh chóng.
\item Đảm bảo đường đi không quá dài để tránh tốn kém năng lượng cho toàn mạng.
\item Đơn giản hóa việc tính toán tại các nút mạng để giảm thời gian truyền tin từ nguồn tới đích, đồng thời cũng tiết kiệm năng lượng cho các nút mạng.
\end{itemize}
Giả thiết bài toán sẽ được trình bày ở chương \ref{sec:4.1}, bao gồm hai phần: mô tả mô hình mạng và giới thiệu các ký hiệu - định nghĩa. Tiếp theo, ở chương \ref{sec:4.2} chúng tôi sẽ trình bày tổng quan về giao thức. Và tiếp theo đó chính là phần mô tả chi tiết về các pha của giao thức bao gồm: chương \ref{sec:4.3} mô tả pha xác định hố, chương \ref{sec:4.4} mô tả pha phát tán thông tin về vùng lõm của hố, cuối cùng, chương \ref{sec:4.5} mô tả phần trọng tâm của giao thức: pha truyền tin (truyền dữ liệu).
 
\section{Giả thiết bài toán}
\label{sec:4.1}
Chúng tôi giả sử rằng ban đầu tất cả các nút mạng đều không biết trước thông tin về hố.
\subsection{Mô hình mạng}
\label{sec:4.1.1}
Mô hình mạng được xét tới trong luận văn này là các mạng có sự xuất hiện của hố (hình 4.1), có nhiều nút nguồn để thu thập thông tin về mục tiêu, và có 1 nút đích duy nhất (một base station) để xử lý dữ liệu đã thu thập được từ các nút nguồn.\\
Giả sử rằng mỗi node đều biết vị trí của nó (sử dụng GPS hoặc các dịch vụ định vị khác \cite{gps}) và vị trí hàng xóm của nó (nhờ quá trình trao đổi gói tin về tọa độ giữa các hàng xóm với nhau). Ngoài ra, nút nguồn biết được vị trí của nút đích.
\begin{figure}[H]
\centering
\includegraphics[width=0.3\textwidth, height=0.3\textwidth]{pics/topo2.png} 
\caption[Mô hình mạng]{Mô hình mạng có sự xuất hiện của hố, các nút hình thoi màu đỏ là nguồn, nút màu xanh nước biển góc trên cùng bên phải của mạng là đích.}
\end{figure}
\subsection{Các ký hiệu và định nghĩa}
\label{sec:4.1.2}
\begin{itemize}
\item \textbf{\textit{Khoảng cách Ơ-clit giữa hai điểm}}: Khoảng cách Ơ-clit giữa 2 điểm $A, B$ chính là độ dài của đoạn thẳng $AB$.
\item \textbf{\textit{Hố định tuyến (hố)}}: \textit{hố định tuyến} là một đa giác có đỉnh là các nút cảm biến $S_{1}, S_{2},... S_{n}$, có các cạnh không cắt nhau tại điểm khác đầu mút của chúng, và thỏa mãn điều kiện:
\begin{itemize}
\item Không có nút cảm biến nào nằm bên trong đa giác này
\item Khoảng cách Ơ-clit giữa $S_{i}$ và $S_{i+1}$ không vượt quá bán kính truyền tin của các nút cảm biến ($\forall i = \overline{1,n}; S_{n+1} \equiv S_{n}$)
\end{itemize}
\item \textbf{\textit{Bao lồi của hố}} (Hình 4.1 (a)): là đa giác lồi có diện tích nhỏ nhất thỏa mãn điều kiện hố nằm hoàn toàn trong đa giác này và có đỉnh là các đỉnh của hố.
\item \textbf{\textit{Vùng lõm của hố}}: là tập hợp các nút liên tiếp trên biên hố nằm giữa 2 đỉnh liên tiếp trên bao lồi của hố (Hình 4.1 (b))(\textit{Lưu ý: chúng tôi không xét những vùng lõm có số đỉnh nhỏ hơn 15}). 
\item \textbf{\textit{Gate-point của vùng lõm}}: là trung điểm của đoạn thẳng nối điểm bắt đầu và kết thúc của vùng lõm (Hình 4.1 (b)).
\item \textbf{\textit{Potential stuck node}}: \textit{potential stuck node} là những điểm nằm trong vùng lõm của hố. Trong trường hợp gói tin dữ liệu xuất phát từ nút nằm ngoài vùng lõm của hố, các potential stuck node sẽ không tham gia vào quá trình truyền tin.
\begin{figure}[H]
\centering
\subfloat[][Bao lồi của hố]{
\includesvg[]{convex_hull_1} 
\label{fig:subfig1}}
\qquad
\subfloat[][Vùng lõm của hố và gate-point tương ứng]{
\includesvg[]{convex_hull} 
\label{fig:subfig2}} \\
\subfloat[][Potential Stuck Node]{
\includesvg[]{convex_hull_2} 
\label{fig:subfig3}}
\caption{Các định nghĩa liên quan tới hố}
\label{fig:globfig}
\end{figure}
\item \textbf{\textit{Khoảng cách ngắn nhất tới gate-point}}: khoảng cách ngắn nhất từ một điểm nằm bên trong vùng lõm $\Re$ tới gate-point của $\Re$ là độ dài đường đi ngắn nhất giữa 2 điểm này mà không cắt $\Re$ (được xác định theo thuật toán tìm đường đi ngắn nhất giữa 2 điểm trong đa giác \cite{shortestpath}). 
\item \textbf{\textit{Next hop}}: \textit{Next hop} là hàng xóm mà nút hiện tại chọn ra để truyền gói tin tới.
\item \textbf{\textit{Stuck node}}: nút mạng mà gói tin đi đến đó và không tìm được next hop.
\end{itemize}
\section{Tổng quan}
\label{sec:4.2}
Giả sử rằng ban đầu tất cả các nút mạng đều không có thông tin về hố. Như đã nêu ở chương \ref{sec:2.1}, khi gói tin dữ liệu đâm vào trong vùng lõm của hố và gặp \textit{stuck node}, nó sẽ phải đi vòng ra khiến đường đi bị kéo dài đáng kể. Trong giao thức HYBRID này, để gói tin dữ liệu có thể tránh hố hiệu quả, trước pha truyền tin dữ liệu, chúng tôi thiết kế 2 pha setup là: \textit{Xác định hố} và \textit{Phát tán thông vùng lõm của hố}. Cụ thể, giao thức HYBRID của chúng tôi bao gồm 3 giao thức nhỏ theo thứ tự lần lượt như sau:
\begin{itemize}
\item \textbf{Giao thức xác định hố}: Thu thập thông tin về tất cả các nút trên biên hố. Mục đích của giao thức xác định hố là để phục vụ cho việc xác định các vùng lõm của hố ở pha sau.
\item \textbf{Giao thức phát tán thông tin vùng lõm của hố}: Trong giao thức này, danh sách các vùng lõm của hố sẽ được xác định và thông tin của mỗi vùng lõm sẽ được phát tán cho những nút nằm bên trong vùng lõm đó. Qua việc phát tán thông tin này, những nút nằm trong vùng lõm của hố sẽ đánh dấu chính nó là \textit{potential stuck node} và tạo thành một vùng cấm ngăn không cho gói tin dữ liệu đi từ ngoài vào bên trong. Bên cạnh đó, những gói tin dữ liệu xuất phát từ bên trong vùng lõm của hố cũng có thể dựa vào thông tin về vùng lõm đã nhận được để thoát hố một cách hiệu quả.
\item \textbf{Giao thức truyền tin}: Ở trong giao thức truyền tin dữ liệu, các nút mạng sẽ sử dụng thông tin về hố đã được xác định ở 2 giao thức đầu để tránh hố. Đồng thời, chúng cũng sử dụng phương pháp Q-learning để học cách tối ưu đường đi, bao gồm: rút ngắn độ dài đường đi, tránh các nút biên hố, chọn các nút mạng có năng lượng còn lại cao, chọn các nút mạng có tần suất nhận tin thấp để làm next hop.  
\end{itemize}
\section{Giao thức xác định hố}
\label{sec:4.3}
Pha phát hiện biên hố được thực hiện theo thuật toán \textit{BOUNDHOLE} \cite{boundhole}, như đã trình bày ở chương 2.1.1. Luật \textit{TENT} được sử dụng để các nút mạng kiểm tra xem mình có khả năng nằm trên biên hố hay không. Những nút mạng có khả năng nằm trên biên hố sẽ tạo gói tin \textit{Hole Boundary Detection (HBD)} và truyền gói tin ấy đi vòng quanh biên hố để thu thập thông tin về các nút nằm trên biên hố. Khi gói tin HBD quay trở lại nút đã tạo ra nó, ta thu được một danh sách đầy đủ các nút trên biên hố. Để tránh việc tất cả các gói HBD đều đi hết một vòng quanh hố, các nút mạng sẽ hủy những gói tin xuất phát từ một nút có tung độ thấp hơn nó. Cuối cùng chỉ có duy nhất một nút có tung độ cao nhất thu được thông tin đầy đủ về hố. Ta gọi nút này là \textit{leader node $ H_{0}$}. 
\section{Giao thức phát tán thông tin vùng lõm của hố}
\label{sec:4.4}
Mục tiêu của giao thức phát tán thông tin hố là làm cho các nút nằm ở trong vùng lõm của hố biết được nó là \textit{potential stuck node} và những hàng xóm nào của nó là \textit{potential stuck node}. Cách làm của chúng tôi đó là sau khi danh sách các nút trên biên hố đã được xác định, danh sách này được gửi đi 1 vòng lần lượt qua các nút trên biên hố, khi gặp 1 nút là điểm bắt đầu của vùng lõm, nút đó sẽ xác định vùng lõm bắt đầu từ nó, sau đó phát tán thông tin về vùng lõm cho tất cả các nút nằm bên trong vùng lõm đó. Nút nào nhận được thông tin phát tán này sẽ đánh dấu chính nó và hàng xóm của nó là \textit{potential stuck node}.\\ 
Giao thức phát tán thông tin vùng lõm của hố bao gồm các bước sau:
\begin{itemize}
\item \textbf{Bước 1 - Gửi gói tin Refresh tới các điểm bắt đầu của vùng lõm:} \textit{leader node $ H_{0}$} xác định các nút bắt đầu của các vùng lõm, sau đó tạo 1 gói tin REFRESH, chèn danh sách các nút trên biên hố và danh sách các nút bắt đầu vùng lõm vào trong gói tin REFRESH. Rồi gửi gói tin REFRESH này đi lần lượt qua từng nút trên biên hố. 
\item \textbf{Bước 2 - Nút bắt đầu vùng lõm xác định vùng lõm của nó:} Khi 1 nút nhận được gói tin REFRESH, nó sẽ lấy ra thông tin danh sách các nút bắt đầu vùng lõm từ trong gói tin này, kiểm tra xem mình có nằm trong danh sách đó không. Nếu có, nó sẽ xác định vùng lõm chứa nó; sau đó tạo 1 gói tin BROADCAST, chèn danh sách các nút nằm trên vùng lõm vào gói tin BROADCAST này và phát tán đi.
\item \textbf{Bước 3 - Phát tán thông tin vùng lõm cho các nút nằm bên trong nó:} Khi 1 nút nhận được gói tin BROADCAST, nó sẽ lấy danh sách các nút trong vùng lõm ra, và kiểm tra xem mình có nằm bên trong vùng lõm này không, nếu có nó tiếp tục phát tán gói tin BROADCAST.
\item \textbf{Bước 4 - Lưu các thông tin cần thiết cho pha định tuyến:} Khi 1 nút nhận được gói tin BROADCAST, ngoài việc xác định xem có phát tán gói tin đi tiếp hay không, nó còn cần phải lưu các thông tin cần thiết để phục vụ cho pha định tuyến. Cụ thể nó sẽ làm các công việc sau:
\begin{itemize}
\item Đánh dấu chính nó và hàng xóm gửi gói tin BROADCAST tới nó là \textit{potential stuck node}
\item Xác định đường đi ngắn nhất từ nó và từ hàng xóm tới \textit{gate-point} của vùng lõm, và lưu lại.
\end{itemize}
\end{itemize}
\begin{center}
\begin{algorithm}[H]
\setcounter{AlgoLine}{0}
\caption{Xác định vùng lõm của hố}\label{cavedetermine}
\KwData{\hspace{4mm}\textit{hole} - list of hole's boundary nodes}
\KwResult{\textit{ID of next hop}}
cavernList = [ ];\\

\For{i in hole} {
	cavern = [ ];\\
	\If{i.isConvexHullBoundary \& !$(i+1)$.isConvexHullBoundary} {
	cavern.append(i++)
	\While{!i.isConvexHullBoundary} {
		cavern.append(i);
		}
	}
}
\Return cavernList
\end{algorithm}
\end{center}
Ở trong pha này chúng tôi sử dụng 2 phương thức: \textit{phương thức xác định vùng lõm của hố} (ở Bước 1, Bước 2)  được trình bày ở \textbf{Algorithm 1} và \textit{phương thức tìm đường đi ngắn nhất trong đa giác} (Bước 4) được giới thiệu trong \cite{shortestpath}.
\section{Giao thức truyền tin}
\label{sec:4.5}
Giao thức truyền tin là phần chính trong giao thức do chúng tôi đề xuất. Giao thức này có nhiệm vụ truyền các gói tin dữ liệu từ nguồn tới đích. \\
Như đã trình bày ở chương \ref{sec:1.4} \textit{Phương hướng và mục tiêu nghiên cứu của đồ án}, giao thức định tuyến mà chúng tôi đề xuất là sự kết hợp giữa định tuyến địa lý và phương pháp học tăng cường, cụ thể là sử dụng thuật toán Q-learning. Trong đó, định tuyến địa lý làm nhiệm vụ đảm bảo được các yếu tố: đường đi của gói tin không bị lặp lại và luôn hướng về đích, gói tin thoát và tránh được hố; và Q-learning sẽ làm nhiệm vụ tối ưu hóa đường định tuyến và đảm bảo cân bằng tải cho các nút mạng (ví dụ như chọn đường đi ngắn, tạo ra sự đa dạng giữa đường định tuyến của các gói tin xuất phát từ cùng một nút nguồn để phân bố tải đều trên toàn mạng...).\\
Dựa theo phương pháp Q-learning, đối với mỗi nút mạng, chúng tôi xây dựng một bảng $Q_{value}$ chứa các $Q_{value}$ của hàng xóm của nó. $Q_{value}$ thể hiện độ tốt của hàng xóm đối với đích. Bảng $Q_{value}$ được cập nhật sau mỗi lần nút mạng gửi gói tin đi. Khi một nút cần truyền gói tin đi, ban đầu nó sẽ sử dụng định tuyến địa lý để xác định ra tập các hàng xóm mà nó có thể truyền gói tin tới, sau đó chọn trong tập các hàng xóm này ra một nút có $Q_{value}$ lớn nhất để làm next hop. Cụ thể về các cơ chế truyền tin và việc sử dụng Q-learning để tối ưu hóa đường định tuyến sẽ được trình bày ở chương \ref{sec:4.5.1} và chương \ref{sec:4.5.2} tương ứng dưới đây.
\subsection{Các cơ chế truyền tin}
\label{sec:4.5.1}
Để phục vụ cho việc định tuyến thoát hố và chuyển hướng khi gặp hố, chúng tôi đã thiết kế 3 cơ chế định tuyến: \textbf{\textit{TOWARD DESTINATION}} (hướng tới đích) được sử dụng bất cứ khi nào có thể; \textbf{\textit{BYPASS HOLE}} (tránh hố) được sử dụng khi không tìm được next hop theo cơ chế \textit{toward destination}; \textbf{\textit{ESCAPE HOLE}} (thoát hố) được sử dụng khi truyền tin ở bên trong vùng lõm của hố.\\
Khi một nút muốn tìm next hop để gửi gói tin tới, trước tiên nó sẽ kiểm tra xem có hàng xóm nào của nó là \textit{potential stuck node} hay không. Có hai trường hợp xảy ra:
\begin{itemize}
\item Nếu tất cả các hàng xóm đều là \textit{potential stuck node}, nghĩa là nút hiện tại nằm bên trong vùng lõm của hố, nó sẽ tìm next hop theo cơ chế Escape Hole để thoát hố. 
\item Nếu tồn tại hàng xóm không phải \textit{potential stuck node}, nghĩa là nút hiện tại không nằm bên trong vùng lõm của hố, trước tiên nó sẽ tìm next hop theo cơ chế Toward Destination. Khi không tìm được next hop theo cơ chế Toward Destination, nút mạng sẽ chuyển sang tìm next hop theo cơ chế Bypass Hole.
\end{itemize}
\begin{center}
\begin{algorithm}[H]
\setcounter{AlgoLine}{0}
\caption{Tìm next hop}\label{findnexthop}
\KwResult{\textit{ID of next hop}}
int nextHop;\\
bool k = checkAllNeighborsIsPotentialStuckNode();\\
\If{k}{
	nextHop $\leftarrow$ findNextHopByHoleEscape();
}
\Else {
	nextHop $\leftarrow$ findNextHopByTowardDestination();\\
	\If{nextHop == NULL} {
		nextHop $\leftarrow$ findNextHopByBypassHole();
	}
}
\Return nexthop.id
\end{algorithm}
\end{center}
Ý tưởng chung của cả ba cơ chế trên đều là lập ra một danh sách các hàng xóm thỏa mãn điều kiện ràng buộc về vị trí địa lý đối với hố và đối với nút đích, sau đó chọn ra từ danh sách này một nút có $Q_{value}$ lớn nhất để làm next hop.
\begin{figure}[H]
    \centering
    \includesvg[width=0.9\textwidth]{routing_path_illusion}
    \caption[Minh họa đường định tuyến của giao thức HYBRID]{Minh họa đường định tuyến của giao thức HYBRID}
    \label{fig:fig4}
\end{figure}
Trong hình trên, hố được thể hiện bởi đa giác màu xanh lục. Nút nguồn \textit{Source} nằm trong vùng lõm của hố, điểm tròn màu cam là \textit{Gate Point} của vùng lõm chứa nút nguồn. Vì \textit{Source} nằm trong vùng lõm của hố nên ban đầu gói tin đi theo cơ chế \textit{Escape Hole} tới \textit{Gate Point}, được thể hiện bởi đường mũi tên màu tím. Khi ra đến \textit{Gate Point}, nút mạng đang giữ gói tin lúc này biết mình không nằm ở trong vùng lõm nữa, nên sẽ cố gắng tìm next hop theo cơ chế \textit{Toward Destination}, nhưng vì không có hàng xóm nào thỏa mãn điều kiện ràng buộc của cơ chế \textit{Toward Destination} nên gói tin được truyền đi theo cơ chế \textit{Bypass Hole}, được thể hiện bởi đường mũi tên màu đỏ. Cuối cùng, gói tin quay trở lại được cơ chế \textit{Toward Destination}, được thể hiện bởi đường mũi tên màu xanh nước biển, và luôn đi theo cơ chế này cho tới đích.\\ \\
Chi tiết về các cơ chế sẽ được trình bày ở dưới đây.\\ \\
\textbf{(i) Cơ chế Toward Destination}\\
Trong cơ chế truyền tin \textit{Toward Destination}, nút mạng sẽ chọn từ các hàng xóm của nó một danh sách $ L $ các hàng xóm thỏa mãn đồng thời ba điều kiện sau:
\noindent
\begin{enumerate}
\item Không phải \textit{potential stuck node}.
\item Có khoảng cách Ơ-clit tới đích gần hơn khoảng Ơ-clit từ nút hiện tại tới đích.
\item Ký hiệu nút hiện tại là $N_{0}$, nút ngay liền trước nút hiện tại trong đường định tuyến là $N_{1}$, và hàng xóm đang xét là $N_{i}$. Thì hàng xóm $N_{i}$ này phải thỏa mãn điều kiện góc $ \widehat{N_{0}N_{1}N_{i}} > threshold $ ( góc $ \widehat{N_{0}N_{1}N_{i}}$ được tính theo góc tuyệt đối, tức là lấy góc nhọn). Trong đó \textit{threshold} là hằng số lớn hơn $90^{o}$ được xác định trước.
\end{enumerate}
Sau đó, nút mạng sẽ chọn từ trong danh sách $ L $ một hàng xóm có $Q_{value}$ lớn nhất để truyền gói tin tới (thuật toán \ref{towarddest}).\\
Điều kiện (1) đảm bảo gói tin không đi vào bên trong vùng lõm của hố, điều kiện (2) đảm bảo gói tin luôn tiến tới gần đích chứ không đi ra xa, và điều kiện (3) nhằm mục đích giảm độ dài đường đi cho gói tin. Ở trong ví dụ hình 4.4, đường mũi tên màu đen biểu thị đường đi của gói tin, $N_{1}$ là nút hiện tại, $ N_{0}$ là nút ngay trước nút hiện tại, nét đứt hình tròn thể hiện bán kính truyền tin của $N_{1}$; $N_{2}, N_{3}$ là các hàng xóm của $N_{1}$. Giả sử $ threshold = 120^{o}$. $N_{2}$ và $N_{3}$ đều thỏa mãn điều kiện 1 và 2 (giả sử các nút mạng trong hình đều ở xa hố), nhưng $N_{2}$ tạo với $N_{0}$ và $N_{1}$ góc $ \alpha = 99.31^{o} < 120^{o}$, không thỏa mãn điều kiện (c), trong khi đó $N_{3}$ tạo với $N_{0}$ và $N_{2}$ góc $ \beta = 167.02^{o} > 120^{o}$, thỏa mãn điều kiện (c). Vậy chỉ có $N_{3}$ được chọn vào trong danh sách L.\\
Nếu không có điều kiện (c), trường hợp sau có thể xảy ra: nút $N_{1}$ chọn $N_{2}$ để làm next hop, sau đó $N_{2}$ chọn $N_{3}$ làm next hop, như vậy đường đi của gói tin sẽ bị kéo dài.
\begin{center}
\begin{figure}[H]
\definecolor{qqwuqq}{rgb}{0.,0.39215686274509803,0.}
\definecolor{ududff}{rgb}{0.30196078431372547,0.30196078431372547,1.}
\begin{tikzpicture}[line cap=round,line join=round,>=triangle 45,x=0.5cm,y=0.5cm]
\clip(1.4998152419125361,4.857055024652394) rectangle (27.648501911686157,16.918037401971983);
\draw [shift={(14.95003964124913,11.881206014983084)},line width=0.4pt,color=qqwuqq,fill=qqwuqq,fill opacity=0.10000000149011612] (0,0) -- (87.51015062273824:0.5960946809218911) arc (87.51015062273824:186.81821457165188:0.5960946809218911) -- cycle;
\draw [shift={(14.95003964124913,11.881206014983084)},line width=0.4pt,color=qqwuqq,fill=qqwuqq,fill opacity=0.10000000149011612] (0,0) -- (-173.18178542834812:0.5960946809218911) arc (-173.18178542834812:-6.160910359962771:0.5960946809218911) -- cycle;
\draw [line width=1.2pt,dash pattern=on 1pt off 1pt on 2pt off 4pt] (14.95003964124913,11.881206014983084) circle (2.cm);
\draw [->,line width=0.8pt] (5.830538104505216,5.652414399320252) -- (9.409512292510374,6.409505092936729);
\draw [->,line width=0.8pt] (9.409512292510374,6.409505092936729) -- (10.682801186319903,9.369041440710228);
\draw [->,line width=0.8pt] (10.682801186319903,9.369041440710228) -- (11.784024013398414,11.502660668174846);
\draw [->,line width=0.8pt] (11.784024013398414,11.502660668174846) -- (14.95003964124913,11.881206014983084);
\draw [line width=0.8pt,dash pattern=on 1pt off 1pt] (15.081772038484548,14.910689946761089)-- (14.95003964124913,11.881206014983084);
\draw [line width=0.8pt,dash pattern=on 1pt off 1pt] (14.95003964124913,11.881206014983084)-- (18.19072649456645,11.531391624932922);
\begin{scriptsize}
\draw [fill=ududff] (5.830538104505216,5.652414399320252) circle (2.5pt);
\draw [fill=ududff] (22.72742585749111,15.219287709564822) circle (2.5pt);
\draw[color=ududff] (22.700916060034462,15.79539241956909) node {$Destination$};
\draw [fill=ududff] (9.409512292510374,6.409505092936729) circle (2.5pt);
\draw [fill=ududff] (10.682801186319903,9.369041440710228) circle (2.5pt);
\draw [fill=ududff] (11.784024013398414,11.502660668174846) circle (2.5pt);
\draw[color=ududff] (11.742708842420365,11.990321373017686) node {$N_0$};
\draw [fill=ududff] (14.95003964124913,11.881206014983084) circle (2.5pt);
\draw[color=ududff] (15.418626041438694,12.328108358873424) node {$N_1$};
\draw [fill=ududff] (15.081772038484548,14.910689946761089) circle (2.5pt);
\draw[color=ududff] (15.259667459859523,15.368191231575066) node {$N_2$};
\draw [fill=ududff] (18.19072649456645,11.531391624932922) circle (2.5pt);
\draw[color=ududff] (18.101052105587204,12.149279954596857) node {$N_3$};
\draw[color=qqwuqq] (14.07741300936444,12.914268128446617) node {$\alpha = 99.31\textrm{\degre}$};
\draw[color=qqwuqq] (14.812596449168105,10.907416036009586) node {$\beta = 167.02\textrm{\degre}$};
\end{scriptsize}
\end{tikzpicture}
\caption{Minh họa cách chọn next hop của cơ chế Toward Destination}
\end{figure}
\end{center}
\begin{center}
\begin{algorithm}[H]
\setcounter{AlgoLine}{0}
\caption{Cơ chế Toward Destination}\label{towarddest}
\KwData{\textit{dest}: coordinate of destination\\\hspace{1.25cm}\textit{prev}: coordinate of previous node on routing path\\\hspace{1.25cm}\textit{this\_node}: coordinate of this node}
\KwResult{\textit{ID of next hop}}
\textit{candidate\_nexthops} = [ ]\\
\For {n in neighbor\_list}{
\If{n.is\_not\_potential\_stuck\_node\ \& distance(n, dest) < distance(this\_node, dest) \& absolute\_angle(prev, this\_node, n) < threshold}{
	\textit{candidate\_nexthops}.append(n)
}
}
\If{candidate\_nexthops is empty}{
\Return  $-1$
}
nexthop = choose node with max \textit{$Q_{value}$} from \textit{candidate\_nexthops}\\
\Return nexthop.id
\end{algorithm}
\end{center}
\begin{figure}[H]
\centering
\includesvg[width=300pt]{greedy1}
\caption[]{Cơ chế Greedy cổ điển vs. cơ chế Toward Destination được đề xuất}
\end{figure}
So sánh với cơ chế Greedy cổ điển, thay vì chọn next hop là hàng xóm có khoảng cách tới đích nhỏ nhất, cơ chế Toward Destination của chúng tôi chọn next hop là hàng xóm có $Q_{value}$ lớn nhất. $Q_{value}$ là giá trị đã được học dựa trên các yếu tố khoảng cách tới đích, khoảng cách tới hố, potential stuck node,... Vì vậy, sau một vài gói tin dữ liệu ban đầu được truyền đi, cơ chế Toward Destination của chúng tôi đã có thể học được cách tránh hố hay từ đầu chứ không đi đâm vào hố như cơ chế Greedy (hình 4.5).\\
\textbf{(ii) Cơ chế BYPASS HOLE}\\
Khi nút mạng không tìm được next hop theo cơ chế \textit{toward destination}, nó sẽ chuyển sang tìm next hop theo cơ chế \textit{bypass hole}. Việc này thường xảy ra khi gói tin gặp hố.\\
Cơ chế Toward Destination luôn tìm hàng xóm gần đích hơn nó để chuyển gói tin tới, nên các nút mạng nằm trên đường định tuyến luôn gần đích hơn các nút trước đó, vì vậy gói tin sẽ không bao giờ đi vòng lại. Tuy nhiên, khi đi theo cơ chế Bypass Hole, để tránh hố, gói tin có thể phải chuyển tiếp gói tin tới nút xa đích hơn nó. Để tránh trường hợp gói tin đi vòng lại khi đi theo cơ chế Bypass Hole, chúng đôi đặt thông tin hướng đi tới đích của gói tin dữ liệu vào header của nó. Có hai hướng đi tới đích của gói tin là \textit{Trái} và \textit{Phải} được định nghĩa như sau (xem Hình 4.4):
\begin{itemize}
\item Ký hiệu nút hiện tại đang có gói tin dữ liệu là C, đích là D. Nếu next hop nằm ở nửa mặt phẳng bên trái của tia CD, gói tin sẽ có hướng đi là bên trái. Nếu next hop nằm ở nửa mặt phẳng bên phải của tia CD, gói tin sẽ có hướng đi là bên phải. Gói tin sẽ luôn giữ hướng đi đã được xác định từ ban đầu cho tới khi quay lại được cơ chế Toward Destination.
\item Về tính toán hình học, những nút nằm ở nửa mặt phẳng bên trái của đường thẳng CD sẽ có tính chất: lấy tâm là C, góc quay ngược chiều kim đồng hồ từ D tới nút ấy nhỏ hơn $180^{o}$; và những nút nằm ở nửa mặt phẳng bên phải của đường thẳng CD có tính chất ngược lại. Ví dụ trong Hình 4.2, $N_{1}$ nằm ở bên trái đường thẳng CD có góc $(D,C,N_{1}) = 60.78^{o} < 180^{o}$, $N_{2}$ nằm ở bên trái đường thẳng CD có góc $(D,C,N_{2}) = 318.75^{o} > 180^{o}$.
\end{itemize}
\begin{figure}[H]
    \centering
    \includesvg[height=0.4\textwidth]{direction}
    \caption{Định nghĩa hướng đi bên phải và bên trái của gói tin}
    \label{fig:fig}
\end{figure}
Bởi vì chỉ khi không đi được theo cơ chế Toward Destination, gói tin mới đi theo cơ chế Bypass Hole, nên trong cơ chế này, nút mạng chỉ chọn từ trong số các hàng xóm xa đích hơn nó để truyền tin tới. Cụ thể, nút mạng sẽ lập ra một danh sách L các hàng xóm của nó thỏa mãn điều kiện:
\begin{enumerate}
\item Không phải \textit{potential stuck node}.
\item Có khoảng cách tới đích xa hơn khoảng cách từ nút hiện tại tới đích.
\item Thuộc nửa mặt phẳng bên trái của tia CD (ký hiệu nút hiện tại là C, nút đích là D) nếu hướng đi của gói tin là bên trái, và thuộc nửa mặt phẳng bên phải của tia CD nếu hướng đi của gói tin là bên phải.
\end{enumerate}
Sau đó, nút mạng sẽ chọn từ trong danh sách $ L $ một hàng xóm có $Q_{value}$ lớn nhất để truyền gói tin tới.\\ \\
\begin{algorithm}[H]
\setcounter{AlgoLine}{0}
\caption{Cơ chế Bypass Hole}\label{bypasshole}
\KwData{\textit{\textbf{dest}}: coordinate of destination;  \textit{\textbf{direction}}: direction of packet toward destination;  \textit{\textbf{this\_node}}: coordinate of this node}
\KwResult{\textit{ID of next hop}}
$L_{1}$ = [ ]; candidate\_nexthops = [ ];\\
\For {n in neighbor\_list}{
\If{n.is\_not\_potential\_stuck\_node\ \& distance(n, dest) >= distance(this\_node, dest)}{
	$L_{1}$.append(n)
}
}
\textcolor{blue}{$//$ \textit{first time switch to Bypass Hole mode}}\\
\If{direction == None} {
	direction = findDirectionToDest($L_{1}$, dest)
}
\Switch{direction} {
\Case{Left} {
	\For{n in $L_{1}$} {
		\If{angle(this\_node, dest, n) < $180^{o}$} {
			candidate\_nexthops.append(n)
		}
	}
}
\Case{Right} {
	\For{n in $L_{1}$} {
		\If{angle(this\_node, dest, n) >= $180^{o}$} {
			candidate\_nexthops.append(n)
		}
	}
}
\Other{
	\Return -1
	}
}
\end{algorithm}
\pagebreak
\begin{algorithm}[H]
\setcounter{AlgoLine}{0}
\caption{Find Direction To Dest}
\KwData{\textit{\textbf{dest}}: coordinate of destination; \textit{\textbf{$L_{1}$}}: list of candidate next hops;}
\KwResult{\textit{direction to destination}}
	neighbor = neighborWithShortestDistanceToDestination($L_{1}$, dest)\\
	\If{isLeftDirection(neighbor, destination)} {
		direction = Left
	} 
	\Else {
		direction = Right
	}
\Return direction
\end{algorithm}
\textbf{(iii) Cơ chế ESCAPE HOLE}\\
Cơ chế \textit{escape hole} được đặc biệt thiết kế cho các trường hợp nút nguồn nằm bên trong vùng lõm của hố.
Khi định tuyến theo cơ chế Escape Hole, một nút sẽ xác định một danh sách L các hàng xóm của nó thỏa mãn điều kiện: \textit{khoảng cách ngắn nhất từ hàng xóm tới gate-point nhỏ hơn khoảng cách ngắn nhất từ nút hiện tại tới gate-point}. Sau đó, từ tập L, nút có $Q_{value}$ cao nhất sẽ được chọn làm nút tiếp theo để gửi gói tin dữ liệu tới.
\begin{center}
\begin{algorithm}[H]
\setcounter{AlgoLine}{0}
\caption{Cơ chế Escape Hole}\label{escapehole}
\KwResult{\textit{ID of next hop}}
\textit{candidate\_nexthops} = [ ]\\
\For {n in neighbor\_list}{
\If{n.shortest\_way\_to\_gate\_point < self.shortest\_way\_to\_gate\_point}{
	append \textit{n} to \textit{candidate\_nexthops}
}
}
\If{candidate\_nexthops is empty}{
\Return  $-1$
}
nexthop = choose node with max \textit{$Q_{value}$} from \textit{candidate\_nexthops}\\
\Return nexthop.id
\end{algorithm}
\end{center}
\subsection{Sử dụng Q-learning để tối ưu hóa đường định tuyến của gói tin}
\label{sec:4.5.2}
Trong chương \ref{sec:4.5.1} ở trên, chúng tôi đã trình bày ba cơ chế định tuyến của giao thức HYBRID. Các cơ chế này đều dựa trên bảng $Q_{value}$ (chứa $Q_{value}$ của các hàng xóm) để tìm ra nút thích hợp nhất làm next hop. Trong chương này chúng tôi sẽ trình bày cách xây dựng và cập nhật bảng $Q_{value}$ cho các nút mạng.\\  \\
Trước tiên, các thành phần của Q-learning được chúng tôi áp dụng vào trong định tuyến như sau:
\begin{itemize}
\item \textbf{\textit{Agent}} (đối tượng): Một \textit{agent} là một gói tin dữ liệu.
\item \textbf{\textit{Environment}} (môi trường): \textit{Environment} là tập hợp tất cả các nút mạng.
\item \textbf{\textit{State}} (trạng thái): Một \textit{state} \textbf{s} là trạng thái gói tin đi tới nút mạng \textbf{\textit{s}}.
\item \textbf{\textit{Action}} (hành động): Một \textit{action} \textbf{a} là sự kiện nút mạng \textbf{a} được chọn làm next hop để gửi gói tin.
\item \textbf{\textit{Reward}} (phần thưởng): \textit{Reward} là một giá trị thực vô hướng được tính dựa trên số liệu về trạng thái của hàng xóm (ví dụ như khoảng cách tới đích, năng lượng còn lại,...) khi gói tin đã được gửi tới hàng xóm đó. 
\item \bm{$Q_{value}$}: $Q_{value}$ của một nút mạng là một giá trị thực vô hướng, thể hiện "độ tốt" của nút mạng ấy đối với đích. Mỗi nút mạng lưu các giá trị $Q_{value}$ của hàng xóm của nó và cập nhật sau mỗi lần gửi tin.
\item \bm{$V_{value}$}: $V_{value} = max(Q_{value})$. $V_{value}$ của một nút mạng là giá trị lớn nhất trong tất cả các $Q_{value}$ của hàng xóm của nó. Khi cập nhật $Q_{value}$ của một hàng xóm ta cần có $V_{value}$ của hàng xóm ấy (theo công thức 4.1), mục đích là để tối ưu hóa $Q_{value}$ dựa trên cả những khả năng chọn next hop ở tương lai, chứ không chỉ tối ưu hóa địa phương dựa trên next hop ngay tiếp theo.
\end{itemize} 
Khi gói tin đang ở trạng thái \textbf{\textit{s}}, nút mạng \textbf{\textit{s}} chọn hàng xóm \textbf{\textit{a}} để gửi gói tin tới. Sau khi \textbf{\textit{s}} gửi gói tin tới \textbf{\textit{a}}, giá trị reward \textbf{\textit{r}} được tính dựa trên các thông tin về hàng xóm \textbf{\textit{a}}. Sau khi reward \textbf{\textit{r}} được tính, $Q_{value}$ của \textbf{\textit{a}} sẽ được cập nhật theo công thức sau:
\begin{equation}
Q(a) \leftarrow (1 - \alpha). Q(a) + \alpha . [r + \gamma . V_{value}(a)]
\end{equation}  
Trong đó, $\alpha$ là learning rate và $\gamma$ là discount factor.\\ \\
Trong giao thức HYBRID, giá trị $Q_{value}$ của hàng xóm a được khởi tạo tại mỗi nút mạng như sau:
\begin{equation}
\text{$Q_{value}$(a)} = \begin{cases}
			-\text{ \textit{Khoảng cách Ơ-clit ngắn nhất tới đích của a}}, \\ \text{\qquad \qquad \qquad \qquad nếu a không phải là potential stuck node} \\
			-\text{\textit{ Khoảng cách ngắn nhất tới gate-point của a}},\\ \text{\qquad \qquad \qquad \qquad nếu a là potential stuck node}
			\end{cases} 
\end{equation}
Với cách khởi tạo này, những nút càng gần đích hoặc càng gần gate-point sẽ có $Q_{value}$ khởi tạo càng lớn, và có xu hướng được chọn làm next hop ngay từ lần gửi gói tin data đầu tiên.\\ \\
Công thức 4.1 được áp dụng cho tất cả các thuật toán Q-learning. Yếu tố quyết định tính hiệu quả của một thuật toán Q-learning là cách xây dựng hàm tính reward \textbf{\textit{r}}. Sau đây chúng tôi sẽ trình bày cách xây dựng hàm tính reward \textbf{\textit{r}}.
Trong việc tối ưu hóa đường định tuyến và đảm bảo cân bằng tải cho các nút mạng, chúng tôi quan tâm tới 4 yếu tố sau:\\ \\
\textbf{(i) Khoảng cách Ơ-clit tới đích của các nút mạng trên đường định tuyến}\\ Khoảng cách tới đích của các nút mạng càng gần, độ dài đường đi tới đích càng giảm. \\ \\
\textbf{(ii) Khoảng cách tới hố của các nút mạng trên đường định tuyến}: $H_{index}$ \\
Khi gặp hố, gói tin có xu hướng không đi được theo cơ chế \textit{Toward Destination} nữa và phải chuyển sang cơ chế \textit{Bypass Hole} để đi vòng qua biên hố. Điều này vừa kéo dài đường định tuyến vừa làm giảm năng lượng của các nút trên biên hố dẫn đến hiện tượng hố bị lan rộng. Vì vậy, trong cơ chế định tuyến của chúng tôi, những nút mạng nằm ở xa hố sẽ được ưu tiên chọn làm next hop để gửi gói tin tới.\\
Chúng tôi ký hiệu chỉ số \textit{\textbf{$H_{index}$}} để thể hiện khoảng cách tới hố của các nút mạng. Mỗi nút mạng cần lưu $H_{index}$ của chính nó và của các hàng xóm của nó. Khi khởi tạo mạng, tất cả các nút mạng sẽ gán $ H_{index} $ của mình bằng $MAX\_INTEGER$. Sau đó, đến pha phát tán thông tin vùng lõm của hố (chương \ref{sec:4.3}), những nút nhận được gói tin Refresh sẽ biết mình nằm trên biên hố, và gán lại $ H_{index} = 0$. Những nút mạng không nằm trên biên hố sẽ cập nhật $H_{index}$ của mình và của hàng xóm khi nhận được gói tin Acknowledgement trong quá trình truyền tin (cách cập nhật $H_{index}$ sẽ được trình bày ở bên dưới).\\ \\
\textbf{(iii) Năng lượng còn lại của các nút mạng: $E_{residual}$}\\
Nút mạng có năng lượng cao sẽ được ưu tiên chọn làm next hop để gửi gói tin tới. \\ \\
\textbf{(iv) Tần suất gói tin đã gửi đi: }\textit{Send Frequency}\\
Để cân bằng năng lượng cho các hàng xóm, mỗi nút mạng sẽ ưu tiên chọn hàng xóm có tần suất gửi tin từ nó tới hàng xóm đó nhỏ nhất trong số tất cả các hàng xóm để làm next hop. Tần suất gửi tin được tính theo công thức:
\begin{flalign} 
\text{Send Frequency} = \varphi * \dfrac{1}{NOW - t} + (1 - \varphi) * \text{Old Send Frequency}
\end{flalign}
Trong công thức 4.2 ở trên, NOW là thời điểm gửi gói tin hiện tại, t là thời điểm gửi gói tin lần gần nhất trước đó; $\varphi$ là hằng số quyết định độ quan trọng của tần suất gửi tin tức thời và tần suất gửi tin trong quá khứ, $\varphi$ lớn thể hiện tần suất tức thời là quan trọng, $\varphi$ nhỏ thể hiện tần suất trong quá khứ là quan trọng.\\ \\ \\
Mỗi nút mạng sau khi nhận được một gói tin dữ liệu, nó cần gửi lại một gói tin Acknowledgement trong đó chứa các thông tin về reward, để nút gửi cập nhật bảng $Q_{value}$. Tuy nhiên vì mạng có rất nhiều cảm biến, trong đó có rất nhiều nút nguồn, mỗi nguồn lại gửi một lượng lớn gói tin dữ liệu tới đích, do đó số lần gói tin dữ liệu được gửi đi là rất lớn. Nếu gói tin Acknowledgement được gửi lại sau mỗi lần nhận gói tin dữ liệu thì không những không đảm bảo được cân bằng tải cho mạng mà còn làm cho các nút mạng bị cạn kiệt năng lượng do gửi quá nhiều gói tin Acknowledgement, thậm chí gây ra xung đột trong mạng khiến các gói tin dữ liệu không thể gửi được tới đích.\\
Vì lý do trên, để tránh tiêu hao năng lượng của các nút mạng và tránh xung đột gói tin, giao thức định tuyến của chúng tôi chỉ thực hiện gửi lại gói tin Acknowledgement theo định kỳ. Cụ thể khi năng lượng còn lại của nút nhận thỏa mãn điều kiện sau thì nó sẽ gửi lại gói tin Acknowledgement cho nút nhận:
\begin{center}
$ \textit{Remaining Energy < Initial Energy * Energy Reduction Threshold}$
\end{center}
Trong 4 yếu tố (i) (ii) (iii) (iv) đã nêu trên, có 2 yếu tố: \textit{Tần suất gói tin đã gửi} và \textit{Khoảng cách Ơ-clit tới đích} có thể được cập nhật ngay tại nút gửi sau khi đã truyền gói tin đi, 2 yếu tố còn lại: \textit{Khoảng cách tới hố của các nút mạng trên đường định tuyến} và \textit{Năng lượng còn lại của các nút mạng} thì cần cập nhật từ gói tin Acknowledgement.\\ \\ Dựa vào quan sát trên, chúng tôi có 2 bước tính reward như sau:
\begin{enumerate}
\item Tính reward ngay sau khi gửi gói tin dữ liệu đi. Giá trị reward sẽ phụ thuộc vào khoảng cách Ơ-clit tới hố của nút nhận và tần suất gửi gói tin tới nút nhận (Send Frequency):
\begin{equation}
r = \begin{cases}
			-R_{stuck}, \text{ nếu không tìm được next hop} \\
			R_{dest}, \text{ nếu next hop là đích} \\
			- \beta_{1} * dist/sum(dist) - (1 - 	\beta_{1}) * \dfrac{1}{1 + e^{-\text{\textit{Send Frequency}}}},\\ \text{\qquad \qquad \qquad \qquad \qquad trong các trường hợp còn lại}
			\end{cases} 
\end{equation}
trong đó: $- R_{stuck}$ là một hằng số âm tương ứng với reward khi gặp một stuck node, $R_{dest}$ là một hằng số dương tương ứng với reward khi gặp đích, \textit{dist} là khoảng cách từ nút nhận tới đích, \textit{sum(dist)} là tổng khoảng cách của các hàng xóm tới đích, \textit{Send Frequency} là tần suất gửi tin tới hàng xóm, đã được định nghĩa ở trên.

\item Tính reward khi nhận được gói tin Acknowledgement từ hàng xóm. Công thức tính của hàm reward như sau:
\begin{equation}
r = \begin{cases}
			-R_{stuck}, \text{ nếu không tìm được next hop} \\
			R_{dest}, \text{ nếu next hop là đích} \\
			- \beta_{2} *(1 - \dfrac{E_{residual}}{E_{initial}}) - (1 - \beta_{2}) * \dfrac{1}{1 + e^{H_{index}}}, \\ \text{\qquad \qquad \qquad trong các trường hợp còn lại}
			\end{cases} 
\end{equation}
trong đó: $- R_{stuck}$ là một hằng số âm tương ứng với reward khi gặp một stuck node, $R_{dest}$ là một hằng số dương tương ứng với reward khi gặp đích, $E_{residual}$ là năng lượng còn lại của nút nhận, $E_{initial}$ là năng lượng khởi tạo của nút nhận, $H_{index}$ là chỉ số thể hiện khoảng cách tới hố của nút nhận (đã được định nghĩa ở trên).\\
Có 2 loại gói tin Acknowledgement là \textit{gói tin Acknowledgement đặc biệt}: được gửi lại từ một stuck node hoặc từ nút đích; và \textit{gói tin Acknowledgement thông thường} được gửi lại từ các nút không phải stuck node cũng không phải nút đích. Các thông tin được gắn vào 2 loại gói tin này như sau:
\begin{table}[h]
\centering
\subfloat[][Gói tin thông thường]{
\footnotesize
\begin{tabular}{ |l|l| }
\hline
\textbf{Field} & \textbf{Type} \\
\hline
\textit{Residual Energy} & double\\
\hline
\textit{$V_{value}$} & double\\
\hline
\textit{$H_{index}$} & int\\
\hline
\end{tabular}
}
\qquad
\subfloat[][Gói tin đặc biệt]{
\footnotesize
\begin{tabular}{ |l|l| }
\hline
\textbf{Field} & \textbf{Type} \\
\hline
\textit{Reward ($R_{stuck}$ hoặc $R_{dest}$)} & double\\
\hline
\textit{$V_{value}$} & double\\
\hline
\end{tabular}
}
\caption{Header của gói tin Acknowledgement}
\end{table}


\end{enumerate}

\chapter{Thí nghiệm và đánh giá hiệu năng của giao thức đề xuất}
\label{sec:5}
\section{Kịch bản thí nghiệm}
\label{sec:5.1}
\begin{figure}[h]
\centering
\subfloat[Topo 1][Topo 1]{
\includegraphics[width=0.3\textwidth, height=0.3\textwidth]{pics/topo_backgrounds/topo1}
\label{fig:subfig1}}
\subfloat[Topo 2][Topo 2]{
\includegraphics[width=0.3\textwidth, height=0.3\textwidth]{pics/topo_backgrounds/topo2}
\label{fig:subfig2}}
\subfloat[Topo 3][Topo 3]{
\includegraphics[width=0.3\textwidth, height=0.3\textwidth]{pics/topo_backgrounds/topo3}
\label{fig:subfig3}}
\caption{Danh sách địa hình mạng lấy từ bản đồ của Google Maps.}
\label{fig:globfig}
\end{figure}
Chúng tôi chọn 3 địa hình thực tế (hình 6.1) lấy từ dữ liệu của Google Maps để làm thực nghiệm. Những địa hình này là vùng có hồ nước, nơi không thể đặt được cảm biến. Khi định tuyến, các hồ nước này trở thành hố định tuyến làm cản trở đường đi của gói tin. Trong các địa hình (a)(b)(c) ở trên, phần màu xanh đậm (topo 2) và màu đen đậm (topo 1, 3) ở chính giữa của các địa hình là các hồ nước.\\
Để chuyển các địa hình trong hình 6.1 thành mô hình mạng, chúng tôi sử dụng công cụ tạo kịch bản mạng và phân tích kết quả thí nghiệm \textbf{WiSSim} \cite{wissim}. Công cụ WiSSim cho phép đưa tệp dưới dạng ảnh vào để làm khung nền cho mạng. Sau khi file ảnh của địa hình thực tế được đưa vào để làm khung nền, chúng tôi tạo mạng có kích thước $ 1000*1000m^{2} $, trong đó 3000 nút mạng được phân bố trong mạng bằng cách chia mạng thành lưới $ 55 * 55 $ ô vuông, và phân bố 1 nút mạng ở vị trí ngẫu nhiên trong mỗi ô vuông ấy. Mật độ của mạng là $ 1  node/18 * 18 m^{2} $. Sau đó nút mạng ứng với vị trí của các hồ nước trong khung nền sẽ được xóa đi để tạo thành hố.\\
Cả ba topo mạng đều được bố trí duy nhất một nút đích (base station) ở biên mạng hoặc góc mạng. Topo 1, topo 2 và topo 3 lần lượt có 44, 51 và 72 nút nguồn tương ứng nằm ở vùng lận cận hố (Hình 6.3). Thời gian mô phỏng thí nghiệm là 3000s. Thông số của các nút mạng được thể hiện ở Bảng 6.1. 

\begin{figure}[H]
\centering
\subfloat[Topo 1][Topo 1]{
\includegraphics[width=0.3\textwidth, height=0.3\textwidth]{pics/topo1.png}
\label{fig:subfig1}}
\subfloat[Topo 2][Topo 2]{
\includegraphics[width=0.3\textwidth, height=0.3\textwidth]{pics/topo2.png}
\label{fig:subfig2}}
\subfloat[Topo 3][Topo 3]{
\includegraphics[width=0.3\textwidth, height=0.3\textwidth]{pics/topo3.png}
\label{fig:subfig3}}
\caption[Vị trí nguồn đích của các topo mạng]{Vị trí nguồn đích của các topo mạng. Các nút màu xanh nhạt là các nút mạng, các vị trí màu đỏ hình thoi là nút nguồn, vị trí màu xanh đậm hình tam giác là đích.}
\label{fig:globfig}
\end{figure}

\begin{table}[H]
\subfloat[][Thông số của một nút mạng]{
\footnotesize
\begin{tabular}{ |l|l| }
\hline
\textbf{Factor} & \textbf{Value} \\
\hline
MAC type & CSMA/CA \\ 
Interface queue model & DropTail \\ 
Transmission of radio & TwoRayGround \\  
Antenna type & OmniAntenna \\  
Node initial energy & 30 J \\  
Queue length & 50 packets \\  
Transmission range & 40 m \\  
Node idle power & 9.6 mW \\  
Node receive power & 45 mW \\  
Node transmit power & 88.5 mW \\ 
Packet sending interval & 10 s \\
Data packet size & 50 bytes\\
\hline
\end{tabular}
}
\qquad
\subfloat[][Thông số của giao thức HYBRID. Các tham số này được xác định dựa trên nhiều lần thí nghiệm và chọn ra bộ số cho kết quả tốt nhất]{
\footnotesize
\begin{tabular}{|l|l|}
\hline
\textbf{Perimeter} & \textbf{Value} \\
\hline
Learning Rate $\alpha$ & 0.9 \\ 
Discount Factor $\gamma$ & 0.5 \\ 
$\beta_{1}$ & 0.4\\
$\beta_{2}$ & 0.6\\
$\varphi \text{ (send frequency factor} $ & 0.5\\
in equation 4.2) & \\
$\text{Energy Reduction Threshold}$ & 0.5\\
$R_{stuck}$ & -15 \\
$R_{dest}$ & 10 \\
\hline
\end{tabular}
}
\caption{Thông số thí nghiệm}
\end{table}

\section{Các tiêu chí đánh giá hiệu năng}
\label{sec:5.2}
Chúng tôi so sánh hiệu năng của giao thức do chúng tôi đề xuất - HYBRID, với hai giao thức đã được đề xuất trước đây: BSMH \cite{bsmh} và EDGR \cite{edgr}, trong đó giao thức BSMH được thí nghiệm với 2 giá trị của $\epsilon$ là $ 0.3 $ và $ 0.8 $, giao thức của chúng tôi có các thông số được trình bày ở Bảng 6.2. Các chỉ số được dùng để đánh giá hiệu năng bao gồm: \textbf{Thời gian sống của mạng} (\textit{Network lifetime}), \textbf{Thời gian gói tin đi từ nguồn tới đích} (\textit{Delay}), \textbf{Tỉ lệ gói tin tới đích thành công} (\textit{Delivery ratio}), \textbf{Năng lượng trung bình để gửi một gói tin} (\textit{Energy consumption per packet}). Ý nghĩa và cách tính của mỗi chỉ số như sau:
\begin{enumerate}
\item \textit{Thời gian sống của mạng}: Chỉ số này được dùng trong hầu hết tất cả các bài báo đề xuất giao thức truyền tin. Có thể thấy, khi có một nút trong mạng chết đi, đặc biệt là nút đó nằm ở vị trí trung tâm của các đường định tuyến, sẽ có rất nhiều gói tin khi được truyền đến nút đó thì bị hủy và không tới được đích, làm giảm hiệu quả truyền tin của mạng. Thời gian sống của mạng được tính từ thời điểm mạng bắt đầu hoạt động cho tới thời điểm nút đầu tiên trong mạng chết đi.
\item \textit{Thời gian gói tin đi từ nguồn tới đích}: Chúng tôi tính trung bình cộng thời gian đi từ nguồn tới đích của tất cả các gói tin tới đích thành công. Thời gian gói tin đi từ nguồn tới đích phụ thuộc vào 3 yếu tố: (i) \textit{số nút mà gói tin đã đi qua} (hay nói cách khác là \textit{độ dài đường đi của gói tin}) - số nút này càng ít, gói tin đi tới đích càng nhanh; (ii) \textit{độ đơn giản của giao thức} - giao thức càng đơn giản, thời gian tính càng nhanh, do đó gói tin sẽ chọn được next hop càng nhanh; (iii) \textit{cân bằng tải của giao thức} - nếu giao thức có độ cân bằng tải không tốt, sẽ có một bộ phận các nút phải nhận nhiều gói tin hơi các nút khác, dẫn đến tình trạng các gói tin được gửi tới những nút đó sẽ phải xếp hàng đợi để chờ các gói tin trước đó được gửi đi, và làm kéo dài thời gian đi tới đích.
\item \textit{Tỷ lệ gói tin tới đích thành công}: Được tính bằng tỷ lệ số gói tin tới đích thành công chia cho số gói tin đã được gửi đi từ nút nguồn.
\item \textit{Năng lượng trung để gửi một gói tin}: Được tính bằng tổng năng lượng đã tiêu hao của tất cả các nút chia cho tổng số gói tin tới đích thành công. Một giao thức định tuyến hiệu quả sẽ có năng lượng tiêu hao của các nút mạng thấp và tổng số gói tin tới đích nhiều.
\end{enumerate}

\section{Phân tích kết quả thí nghiệm}
\label{sec:5.3}
\subsection{Thời gian sống của mạng}
Kết quả thí nghiệm thời gian sống của mạng (network lifetime) được thể hiện ở Hình 5.1. Có thể thấy network lifetime của giao thức HYBRID do chúng tôi đề xuất đạt kết quả cao (lớn hơn 1.2 lần so với giao thức BSMH ở Topo 1, 2 và lớn hơn 1.5 lần so với giao thức EDGR ở Topo 3). Đồng thời, network lifetime của giao thức HYBRID cũng ổn định ở cả 3 Topo mạng. Trong khi đó giao thức BSMH đạt kết quả cao hơn EDGR ở Topo 1 và Topo 2 nhưng lại thấp hơn EDGR ở Topo 3. Giao thức BSMH xác định sẵn các đích tạm thời và cho các gói tin đi theo greedy đến cách đích tạm thời ấy. Đường định tuyến của giao thức này chỉ đảm bảo khác nhau qua các lần đi (nhờ thay đổi tỷ lệ vị tự) mà không quan tâm đến năng lượng còn lại của các nút mạng. Giao thức EDGR có quan tâm đến năng lượng còn lại của các nút mạng nhưng lại chỉ tối ưu địa phương, tức là chỉ đảm bảo cân bằng tải giữa các hàng xóm của một nút mạng, và vẫn cho gói tin đi theo greedy tới các đích tạm thời. Khác với hai giao thức này, giao thức của chúng tôi sử dụng Q-learning để thực hiện back propagation về trạng thái năng lượng của cả những nút mạng ở xa, đồng thời nút mạng sẽ liên tục học và cập nhật được trạng thái về năng lượng của lẫn nhau, vì vậy cân bằng tải của HYBRID tốt hơn, dẫn đến network lifetime được kéo dài. Ngoài ra, việc tính toán phức tạp và phát tán nhiều thông tin về hố (của giao thức BSMH) hay gửi gói tin BURST (của giao thức EDGR) cũng làm tiêu hao năng lượng của các nút mạng.
\begin{figure}[H]
\centering
\includegraphics[scale=0.6]{pics/network_lifetime.png}
\caption{Kết quả thí nghiệm thời gian sống của mạng}
\end{figure}
\subsection{Thời gian gói tin đi từ nguồn tới đích}
\begin{figure}[H]
\centering
\includegraphics[scale=0.6]{pics/average_delay.png}
\caption{Kết quả thí nghiệm trung bình thời gian gói tin đi từ nguồn tới đích}
\end{figure}
Qua kết quả trên, giao thức HYBRID cho thấy hiệu quả trong tốc độ truyền tin. Thời gian truyền tin từ nguồn tới đích của HYBRID chỉ bằng khoảng 0.65 lần thời gian truyền tin của BSMH và khoảng 0.26 lần thời gian truyền tin của EDGR. Như đã đề cập ở trên, tốc độ truyền tin phụ thuộc vào 3 yếu tố là: độ dài đường đi, độ phức tạp tính toán, và cân bằng tải. Khi đánh giá tốc độ truyền tin, chúng tôi cũng đã thí nghiệm đo kết quả về độ dài đường đi của ba giao thức, được thể hiện ở hai chỉ số \textit{Average Stretch} và \textit{Max Stretch} được thể hiện ở hai đồ thị dưới đây:
\begin{figure}[H]
\centering
\subfloat[][Average Stretch]{
\includegraphics[width=0.4\textwidth, height=0.3\textwidth]{pics/average_stretch.png}
\label{fig:subfig1}}
\subfloat[][Max Stretch]{
\includegraphics[width=0.4\textwidth, height=0.3\textwidth]{pics/max_stretch.png}
\label{fig:subfig2}}
\caption{Kết quả thí nghiệm về độ dài đường định tuyến}
\end{figure}
Nhìn vào đồ thị trên ta có thể thấy rằng độ dài đường đi của giao thức HYBRID không tốt hơn độ dài đường đi của giao thức EDGR. Cụ thể là ở Topo 1 và Topo 2 thì HYBRID tốt hơn EDGR nhưng tỷ lệ đường đi giữa 2 giao thức chỉ rơi vào khoảng 0.8, và Topo 3 thì HYBRID lại có độ dài đường đi gấp khoảng 1.2 lần độ dài đường đi của EDGR. Tuy nhiên, tốc độ truyền tin của HYBRID vẫn tốt hơn EDGR rất nhiều bởi HYBRID đã làm tốt được 2 yếu tố là: độ đơn giản tính toán và cân bằng tải. Xét về BSMH, BSMH luôn cho kết quả về độ dài đường đi rất tốt ở cả 3 topo, tuy nhiên giao thức này có độ phức tạp tính toán khá lớn, dẫn đến tốc độ truyền tin không tốt.
\subsection{Tỷ lệ gói tin tới đích thành công}
\label{sec:5.3.3}
\begin{figure}[H]
\centering
\includegraphics[scale=0.6]{pics/delivery_ratio.png}
\caption{Kết quả thí nghiệm tỷ lệ gói tin tới đích thành công}
\end{figure}
Tỷ lệ gói tin tới đích thành công của cả ba giao thức đều đạt xấp xỉ 100\% ở cả ba topo mạng. Ở Topo 2, tỷ lệ gửi tin thành công của HYBRID thấp hơn một chút so với 2 giao thức còn lại (99.33\%). Bởi các đích tạm thời được xác định trước khi truyền tin và gắn vào header của gói tin dữ liệu, hai giao thức BSMH và EDGR đều được chứng minh dựa trên lý thuyết là luôn tìm được next hop. Vì muốn tiết kiệm năng lượng và thời gian ở khâu tính toán đường đi trước khi truyền tin, đồng thời giảm lượng thông tin mà gói tin dữ liệu phải mang đi, giao thức HYBRID của chúng tôi chỉ định tuyến dựa trên kinh nghiệm trong quá khứ và thông tin của các nút hàng xóm. Tuy không có chứng minh lý thuyết về việc đảm bảo tỷ lệ truyền tin, kết quả thực nghiệm cho thấy HYBRID có tỷ lệ truyền tin thành công trên 99\% ở cả 3 topo mạng.
\subsection{Năng lượng trung bình để gửi một gói tin}
\begin{figure}[H]
\centering
\includegraphics[scale=0.6]{pics/energy_per_packet.png}
\caption{Kết quả thí nghiệm năng lượng tiêu hao trung bình cho mỗi gói tin gửi tới đích}
\end{figure}
Giao thức HYBRID cho kết quả tốt ở cả ba topo mạng, BSMH cho kết quả tốt thứ 2, và giao thức EDGR luôn cho kết quả tồi nhất. Nhìn vào kết quả số gói tin tới đích thành công ở chương \ref{sec:5.3.3}, chúng ta kết luận được rằng số lượng gói tin tới đích thành công của cả ba giao thức là tương đương nhau. Vì vậy năng lượng tiêu hao trung bình cho mỗi gói tin gửi tới đích cao hay thấp sẽ phụ thuộc vào 2 yếu tố: \textit{Tổng năng lượng tiêu hao của mỗi nút mạng} và \textit{Số lượng nút mạng đã tham gia vào quá trình định tuyến}. Nhìn vào kết quả thí nghiệm về độ dài đường đi ở Hình 5.7, ta có thể nói rằng EDGR luôn có năng lượng tiêu hao trung bình của mỗi gói tin tồi nhất vì đường đi của giao thức này dài nhất. Độ dài đường đi của HYBRID lớn hơn độ dài đường đi của BSMH chứng tỏ có nhiều nút mạng tham gia vào quá trình định tuyến hơn, nhưng nhờ quá trình tính toán đường đi đơn giản và dung lượng thông tin chứa trong header của gói tin ít hơn (cụ thể, dựa vào các thông tin chứa trong header, chúng tôi đã tính ra dung lượng thông tin mà gói tin dữ liệu của mỗi giao thức HYBRID, BSMH, EDGR cần truyền đi lần lượt là: 24 bytes, 300 bytes, 308 bytes), HYBRID đã tiết kiệm được năng lượng truyền và nhận tin cho mỗi nút mạng. 
\subsection{Tổng reward trung bình theo thời gian của các gói tin trong giao thức HYBRID}
\begin{figure}[h]
\centering
\subfloat[][Topo 1]{
\includegraphics[width=0.5\textwidth, height=0.35\textwidth]{pics/SumReward1.png}
\label{fig:subfig1}}
\subfloat[][Topo 2]{
\includegraphics[width=0.5\textwidth, height=0.35\textwidth]{pics/SumReward2.png}
\label{fig:subfig2}}
\\
\subfloat[][Topo 3]{
\includegraphics[width=0.5\textwidth, height=0.35\textwidth]{pics/SumReward3.png}
\label{fig:subfig3}}
\caption{Trung bình của tổng Reward của các gói tin dữ liệu theo thời gian của từng topo.}
\label{fig:globfig}
\end{figure}
Theo định nghĩa về học tăng cường ở chương \ref{sec:3.1}, mục tiêu của quá trình học có trạng thái kết thúc là tối ưu lượng $R = r_{1} + r_{2} + ... + r_{n}$. Khi áp dụng học tăng cường vào định tuyến, chúng tôi coi trạng thái kết thúc của gói tin dữ liệu là khi nó được truyền tới đích. Chúng tôi đánh giá lượng $R = r_{1} + r_{2} + ... + r_{n}$ bằng cách đo tổng reward được gửi về khi gói tin đi hết một lượt từ nguồn tới đích của các gói tin xuất phát từ các nguồn khác nhau, rồi lấy trung bình của các kết quả ấy. Ở cả 3 topo, với thời gian mô phỏng là 3000s, có tất cả 175 gói tin dữ liệu được gửi đi từ mỗi nút nguồn. Trung bình của tổng reward của các gói tin dữ liệu theo thời gian được thể hiện ở Hình 5.7. Qua đồ thị này, chúng ta có thể thấy thời gian để giá trị reward trở nên ổn định, còn gọi là thời gian cần thiết để xây dựng được một bảng $Q_{value}$ ổn định hay \textit{convergence time} rơi vào khoảng 20 gói tin đầu tiên. Điều này cho thấy rằng, chỉ sau khoảng 10\% thời gian đầu tiên, các gói tin đã có thể đi theo đường đi "tối ưu" từ nguồn tới đích. 
\chapter{Kết luận}
\label{sec:6}
\section{Kết quả đạt được}
\label{sec:6.1}
Dựa trên các ưu điểm và nhược điểm của các giao thức định tuyến địa lý tránh hố trước đây, trong đồ án này, chúng tôi đã đề xuất giao thức định tuyến tránh hố HYBRID kết hợp giữa định tuyến địa lý và phương pháp Q-learning. Kết quả thí nghiệm cho thấy:
\begin{itemize} 
\item Giao thức của chúng tôi đã kéo dài thời gian sống của mạng thêm 20\% so với các giao thức trước đây.
\item Tốc độ truyền tin của giao thức do chúng tôi đề xuất nhanh gấp 1.5 lần so với các giao thức trước đây.
\item Tỷ lệ truyền tin tới đích thành công đạt hơn 99\%.
\end{itemize}
Với kết quả trên, chúng tôi nhận định rằng áp dụng học tăng cường vào định tuyến trong mạng cảm biến không dây là một hướng tiếp cận phù hợp. Và trong quá trình nghiên cứu, tiến hành cài đặt thí nghiệm, chúng tôi nhận ra còn có nhiều hướng để mở rộng và cải tiến việc áp dụng học tăng cường, cụ thể chúng tôi sẽ trình bày ở chương \ref{sec:6.2} dưới đây.
\section{Hướng phát triển}
\label{sec:6.2}
Chúng tôi có hai hướng phát triển giao thức như sau:
\begin{enumerate}
\item Tuy giao thức HYBRID đã được thiết kế để gói tin có thể định tuyến được trong các địa hình mạng có một hay nhiều hố, nhưng trong khuôn khổ của đồ án này, chúng tôi mới chỉ thí nghiệm đối với những topo mạng có một hố duy nhất. Để giao thức được toàn diện, chúng tôi hướng tới việc thí nghiệm đánh giá và tối ưu giao thức cho các trường hợp mạng \textit{có nhiều hố, hoặc địa hình hố thay đổi theo thời gian}.
\item Phương pháp Q-learning thuộc nhóm thuật toán \textit{Tabular Solution Methods}, tức là chỉ phù hợp với những đối tượng có không gian của tập \textit{action} (hành động) và tập \textit{state} (trạng thái) là nhỏ. Vì hiện tại chúng tôi hướng tới các kịch bản mạng có số ít nút đích (ít base station) nên chúng tôi đã chọn lựa phương pháp Q-learning. Trong tương lai, để giao thức có thể hoạt động tốt trong trường hợp mạng có rất nhiều nút đích, chúng tôi sẽ thay đổi cách sử dụng phương pháp học tăng cường, ví dụ như chuyển sang dùng nhóm thuật toán \textit{Approximation Solution Methods} thay vì \textit{Tabular Solution Methods}.
\end{enumerate}

\begin{thebibliography}{100}
\bibitem{wikisensor}
\url{https://vi.wikipedia.org/wiki/C%E1%BA%A3m_bi%E1%BA%BFn}

\bibitem{military1}
S. H. Lee, S. Lee, H. Song, and H. S. Lee, “Wireless sensor network design for
tactical military applications : Remote large-scale environments,” in \textit{MILCOM
2009 - 2009 IEEE Military Communications Conference}, pp. 1–7, Oct 2009.

\bibitem{military2}
E. Onur, C. Ersoy, H. Delic, and L. Akarun, “Surveillance wireless sensor net-
works: Deployment quality analysis,” \textit{IEEE Network}, vol. 21, pp. 48–53, Novem-
ber 2007.

\bibitem{disaster1}
D. Chen, Z. Liu, L. Wang, M. Dou, J. Chen, and H. Li, “Natural disaster moni-
toring with wireless sensor networks: A case study of data-intensive applications
upon low-cost scalable systems,” \textit{Mob. Netw. Appl.}, vol. 18, pp. 651–663, Oct.
2013.

\bibitem{disaster2}
M. Erdelj, M. Król, and E. Natalizio, “Wireless sensor networks and multi-uav
systems for natural disaster management,” \textit{Computer Networks}, vol. 124, pp. 72
– 86, 2017.

\bibitem{disaster3}
G. Han, X. Yang, L. Liu, M. Guizani, and W. Zhang, “A disaster management-
oriented path planning for mobile anchor node-based localization in wireless sen-
sor networks,” \textit{IEEE Transactions on Emerging Topics in Computing}, pp. 1–1,
2017.

\bibitem{disaster4}
A. S. Bhosle and L. M. Gavhane, “Forest disaster management with wireless sen-
sor network,” in \textit{2016 International Conference on Electrical, Electronics, and
Optimization Techniques (ICEEOT)}, pp. 287–289, March 2016.

\bibitem{disaster5}
E. Cayirci and T. Coplu, “Sendrom: Sensor networks for disaster relief operations
management,” \textit{Wireless Networks}, vol. 13, pp. 409–423, Jun 2007.

\bibitem{agri1}
A. ur Rehman, A. Z. Abbasi, N. Islam, and Z. A. Shaikh, “A review of wire-
less sensors and networks’ applications in agriculture,” \textit{Computer Standards \&
Interfaces}, vol. 36, no. 2, pp. 263 – 270, 2014.

\bibitem{agri2}
F. Ingelrest, G. Barrenetxea, G. Schaefer, M. Vetterli, O. Couach, and M. Par-
lange, “Sensorscope: Application-specific sensor network for environmental
monitoring,” \textit{ACM Transaction on Sensor Network}, vol. 6, pp. 17:1–17:32, Mar.
2010..

\bibitem{agri3}
T. Ojha, S. Misra, and N. S. Raghuwanshi, “Wireless sensor networks for agri-
culture: The state-of-the-art in practice and future challenges,” \textit{Computers and Electronics in Agriculture}, vol. 118, pp. 66 – 84, 2015.

\bibitem{agri4}
X. Shi, L. Fan, J. Xia, Z. Tang, and H. Li, “An environment monitoring system
for precise agriculture based on wireless sensor networks,” in \textit{2011 Seventh International Conference on Mobile Ad-hoc and Sensor Networks(MSN)}, vol. 00,
pp. 28–35, 12 2011.

\bibitem{agri5}
Y. Zhu, J. Song, and F. Dong, “Applications of wireless sensor network in the
agriculture environment monitoring,” \textit{Procedia Engineering}, vol. 16, pp. 608 –
614, 2011.

\bibitem{health1}
M. M. Baig and H. Gholamhosseini, “Smart health monitoring systems: An
overview of design and modeling,” \textit{Journal of Medical Systems}, vol. 37, pp. 98 –
98, Jan 2013.

\bibitem{health2}
M. R. Yuce, “Implementation of wireless body area networks for healthcare sys-
tems,” \textit{Sensors and Actuators A: Physical}, vol. 162, no. 1, pp. 116 – 129, 2010.

\bibitem{health3}
H. Alemdar and C. Ersoy, “Wireless sensor networks for healthcare: A survey,”
\textit{Computer Networks}, vol. 54, no. 15, pp. 2688 – 2710, 2010.

\bibitem{sensorenergy}
I. F. Akyildiz, W. Su, Y. Sankarasubramaniam, and E. Cayirci, “Wireless sensor
networks: a survey,” \textit{Computer Networks}, vol. 38, pp. 393–422, 2002.

\bibitem{vhr}
Phi-Le Nguyen, Yusheng Ji, Khanh Le, Thanh-Hung Nguyen, "Load balanced and constant stretch routing in the vicinity of holes in WSNs", 15th IEEE \textit{Annual Consumer Communications \& Networking Conference} (CCNC), 2018.

\bibitem{bsmh}
Phi-Le Nguyen, Yusheng Ji, Khanh Le, Thanh-Hung Nguyen, "Routing in the Vicinity of Multiple Holes in WSNs," \textit{2018 5th International Conference on Information and Communication Technologies for Disaster Management (ICT-DM)}, 2018.

\bibitem{edgr}
Haojun Huang, Hao Yin, Geyong Min, Junbao Zhang, Yulei Wu, Xu Zhang, "Energy-Aware Dual-Path Geographic Routing to Bypass Routing Holes in Wireless Sensor Networks", \textit{IEEE Transactions on Mobile Computing}, vol. 17, issue 6, 2018.

\bibitem{holeplastic}
Fucai Yu, Shengli Pan, and Guangmin Hu, "Hole Plastic Scheme for Geographic Routing in Wireless Sensor Networks", \textit{IEEE ICC - Ad-hoc and Sensor Networking Symposium}, 2015.

\bibitem{gps}
N. Bulusu, J. Heidemann, and D. Estrin, “Gps-less low-cost outdoor localization
for very small devices,” \textit{IEEE Personal Communications}, vol. 7, pp. 28–34, Oct
2000.

\bibitem{boundhole}
Q. Fang, J. Gao, and L. J. Guibas, “Locating and bypassing routing holes in sensor networks,” in \textit{Proc. of the 20th Annual Joint Conference of the IEEE Computer and Communications Societies}, INFOCOM’04, vol. 4, pp. 2458–2468 vol.4, March 2004.

\bibitem{rollingball}
Wen-Jiunn Liu and Kai-Ten Feng, "Greedy Routing with Anti-Void Traversal for Wireless Sensor Networks", IEEE \textit{TRANSACTIONS ON MOBILE COMPUTING}, VOL.8, NO.7, JULY 2009.

\bibitem{bcp}
Zhiping Kang, Honglin Yu, and Qingyu Xiong, "Detection and Recovery of Coverage
Holes in Wireless Sensor Networks", 2013.

\bibitem{grid}
Phi Le Nguyen, Van Khanh Nguyen, "On hole approximation algorithms in wireless sensor networks", \textit{Journal of Computer Science and Cybernetics},  Vol.30, No.4, 2014.

\bibitem{gpsr}
Brad Karp, H. T. Kung, "GPSR: Greedy Perimeter Stateless Routing for Wireless
Networks", \textit{Proceeding MobiCom \'00 Proceedings of the 6th annual international conference on Mobile computing and networking}, Pages 243-254, 2000.

\bibitem{wissim}
\url{http://sedic.soict.hust.edu.vn/wissim/.}

\bibitem{ns2}
\url{https://www.isi.edu/nsnam/ns/}

\bibitem{shortestpath}
D.T. Lee, F.P. Preparata, "Euclidean shortest paths in the presence of rectilinear barriers", \textit{Networks}, vol. 14, no. 3, pp. 393-410, 1984.

\bibitem{subar}
S. Subramanian, S. Shakkottai, and P. Gupta, “On optimal geographic routing
in wireless networks with holes and non-uniform traffic,” in Proc. of the 26th
IEEE \textit{International Conference on Computer Communications}, INFOCOM’07,
pp. 1019–1027, May 2007.

\bibitem{hexagon}
M. Choi, H. Choo, "Bypassing Hole Scheme Using Observer Packets for Geographic Routing in WSNs", \textit{The International Conference on Information Networking}, 2011 (ICOIN2011).

\bibitem{bedh}
F. Yu et al, "Efficient Hole Detour Scheme for Geographic Routing in Wireless Sensor Networks", In Proc. of the 67th \textit{IEEE Vehicular Technology Conference}, VTC'08, pages 153-157, 2008.

\bibitem{elipse}
Y. Tian et al., "Energy-Efficient Data Dissemination Protocol for Detouring Routing Holes in Wireless Sensor Networks". In Proc. of IEEE Intl. Conf. on \textit{Communications}, ICC'08, pages 2322-2326, 2008.

\bibitem{octagon}
Phi-Le Nguyen, Duc-Trong Nguyen, Khanh-Van Nguyen, "Load balanced routing with constant stretch for wireless sensor network with holes", IEEE Ninth International Conference on Intelligent Sensors, \textit{Sensor Networks and Information Processing} (ISSNIP), 2014.

\bibitem{qlearningwsn}
Shaoqiang Dong, Prathima Agrawal and Krishna Sivalingam, "Reinforcement Learning Based Geographic Routing Protocol
for UWB Wireless Sensor Network", \textit{Communications Society subject matter experts for publication} in the IEEE GLOBECOM 2007.

\bibitem{adaptiveqlearning}
Ping Wang, Ting Wang, "Adaptive Routing for Sensor Networks using Reinforcement Learning", Proceedings of The Sixth IEEE \textit{International Conference on Computer and Information Technology} (CIT'06), 2006.

\bibitem{qelar}
Tiansi Hu, Yunsi Fei, "QELAR: A Machine-Learning-Based Adaptive Routing Protocol for Energy-Efficient and Lifetime-Extended Underwater Sensor Networks", Published by the IEEE CS, CASS, \textit{ComSoc}, 2010.
\end{thebibliography}
\end{document}
